\documentclass[11pt,a4paper, final]{moderncv}
\usepackage{color}
\usepackage{fontspec}
\usepackage{url}
\moderncvtheme{classic}
\definecolor{color1}{RGB}{1, 52, 64}
\usepackage[scale=0.95]{geometry}
\setlength{\hintscolumnwidth}{3.5cm} 
\AtBeginDocument{\recomputelengths} 
\usepackage{xunicode}
\usepackage{xltxtra}
\usepackage[utf8]{inputenc}
\usepackage{lipsum} 
\usepackage{tikz}
\usepackage{arydshln}
\usepackage{nccrules}
\usepackage{verbatim}
\setmainfont{Times New Roman}
\AfterPreamble{\hypersetup{
  pdfcreator={XeLaTeX},
  pdftitle={Fictional CV of Hai Jiang}
}}
\usepackage{fontawesome5}
%%%%%%%%%%%%%%%%%%%%%%%%%%%%%%%%%%%%%%%%%%%%%%%%%%%%%%%%%%%%%%%%%%%%%%%%%%%%%%%%%%%%%%%%%%%%%%%%%%%%%%%%%%%%%%%%%%%%%
%%%%%%%%%%%%%%%%%%%%%%%%%%%%%%%%%%%%%%%%%%%%%%%%%%%%%%%%%%%%%%%%%%%%%%%%%%%%%%%%%%%%%%%%%%%%%%%%%%%%%%%%%%%%%%%%%%%%%
%%%%%%%%%%%%%%%%%%%%%%%%%%%%%%%%%%%%%%%%%%%%%%%%%%%%%%%%%%%%%%%%%%%%%%%%%%%%%%%%%%%%%%%%%%%%%%%%%%%%%%%%%%%%%%%%%%%%%
%%%%%%%%%%%%%%%%%%%%%%%%%%%%%%%%%%%%%%%%%%%%%%%%%%%%%%%%%%%%%%%%%%%%%%%%%%%%%%%%%%%%%%%%%%%%%%%%%%%%%%%%%%%%%%%%%%%%%
%%%%%%%%%%%%%%%%%%%%%%%%%%%%%%%%%%%%%%%%%%%%%%%%%%%%%%%%%%%%%%%%%%%%%%%%%%%%%%%%%%%%%%%%%%%%%%%%%%%%%%%%%%%%%%%%%%%%%
\firstname{Hai}
\familyname{Jiang}
\address{Shenzhen, Guangdong, China}{}
\extrainfo{
	\faPhone\hspace{0.5em}+86 19867713757\\
	{\small\faEnvelope}\hspace{0.5em}jiangh14@lzu.edu.cn\\
}
\newcommand{\spacesection}{\vspace{0.4cm}}
\newcommand{\spacesubsection}{\vspace{0.2cm}}
%===========================
\begin{document}
\recipient{Admission Committee}
{Leibniz Information Centre for Science and Technology (TIB)\\
Open Research Knowledge Graph (ORKG) Program\\
% Science and Technology\\
% Logistics and Supply Chain Management
}
\date{\today}
\opening{Dear Members of the Admissions Committee,}
\closing{Sincerely,}
%%%%%%%%%%%%%%%%%%%%%%%%%%%%%%%%%%%%%%%%%%%%%%%%%%%%%%%%%%%%%%%%%%%%%%%%%%%%%%%%%%%%%%%%%%%%%%%%%%%%%%%%%%%%%%%%%%%%%
%%%%%%%%%%%%%%%%%%%%%%%%%%%%%%%%%%%%%%%%%%%%%%%%%%%%%%%%%%%%%%%%%%%%%%%%%%%%%%%%%%%%%%%%%%%%%%%%%%%%%%%%%%%%%%%%%%%%%
%%%%%%%%%%%%%%%%%%%%%%%%%%%%%%%%%%%%%%%%%%%%%%%%%%%%%%%%%%%%%%%%%%%%%%%%%%%%%%%%%%%%%%%%%%%%%%%%%%%%%%%%%%%%%%%%%%%%%
%%%%%%%%%%%%%%%%%%%%%%%%%%%%%%%%%%%%%%%%%%%%%%%%%%%%%%%%%%%%%%%%%%%%%%%%%%%%%%%%%%%%%%%%%%%%%%%%%%%%%%%%%%%%%%%%%%%%%
%%%%%%%%%%%%%%%%%%%%%%%%%%%%%%%%%%%%%%%%%%%%%%%%%%%%%%%%%%%%%%%%%%%%%%%%%%%%%%%%%%%%%%%%%%%%%%%%%%%%%%%%%%%%%%%%%%%%%
\makelettertitle
\thispagestyle{empty}
\pagestyle{empty}
I am writing to express my enthusiastic interest in the PhD position in ORKG at TIB. 
With a background in computational mathematics, hands-on experience in AI-driven research, 
and a passion for bridging technical innovation with structured knowledge systems, 
I am eager to contribute to TIB's mission of advancing scholarly information infrastructure 
through interpretable and scalable AI solutions.

% \textbf{Academic Background \& Research Expertise}\\
During my \textbf{Master's in Computational Mathematics} (2019-2022), 
I specialized in \textbf{Generative Adversarial Networks (GANs)}, 
focusing on cross-domain style transfer. 
My thesis, \emph{``GANs for Personal Style Imitation of Chinese Handwritten Characters,''} 
involved developing an end-to-end CycleGAN-based framework to replicate the calligraphic style of Shiling Shen Chern. 
By optimizing adversarial training and domain-specific preprocessing, 
I achieved \textbf{85\% visual similarity} across 220 characters, surpassing baseline models by 10\%. 
This project honed my expertise in structured data transformation and cross-domain adaptation—skills directly applicable to knowledge graph design, where heterogeneous data integration and semantic mapping are critical.

Post-graduation, I joined Sun Yat-sen University's \textbf{Computational Medical Imaging Lab}, 
where I led AI-driven projects with clinical impact. 
For instance, I developed a \textbf{multi-task learning model} to 
classify Placenta Accreta Spectrum Disorder severity using T2-WI MRI images, achieving an \textbf{AUC of 0.80}. 
In a separate project, 
I designed a CNN-based system to predict breast cancer metastasis in Sentinel Lymph Nodes via dual-energy CT scans, 
attaining an \textbf{AUC of 0.85} in cross-validation. 
Beyond algorithmic development, I curated heterogeneous DICOM datasets and deployed models on hospital servers, 
demonstrating my ability to manage complex data pipelines—a skill vital for building robust knowledge infrastructures. 
These efforts resulted in two first-authored papers (one accepted at \textbf{ISBI 2025}, one under revision).

% \textbf{Alignment}\\
Your program's focus on \textbf{Knowledge Graphs} deeply resonates with my goal 
to enhance scholarly information systems through mathematically rigorous AI.  
I am particularly inspired by \textbf{Prof. Sören Auer's work} 
on integrating mathematical reasoning into Large Language Models (LLMs), 
which aligns with my interest in developing AI frameworks that balance theoretical foundations with real-world applicability. 
For instance, my CycleGANs project required structuring unstructured handwriting data into transferable styles—a precursor to knowledge graph tasks like entity alignment and semantic enrichment. Similarly, my medical imaging work involved collaborating with clinicians to ensure models addressed practical needs, mirroring the human-AI collaboration essential for designing user-centric knowledge systems.

TIB's \textbf{ORKG} infrastructure and partnerships with industry leaders 
provide an unparalleled environment to address challenges in knowledge representation, 
graph neural networks (GNNs), and explainable AI. 
I am keen to contribute to projects such as: 
(1) Enhancing knowledge graph embeddings using GNNs to improve semantic relationships in scholarly data; 
(2) Developing LLM-driven tools for automated knowledge extraction and curation from multidisciplinary research articles; 
(3) Designing evaluation frameworks to ensure mathematical explainability in AI-driven knowledge systems.

% \textbf{Long-Term Goals}\\
Long-term, I aspire to pioneer interpretable AI systems that harmonize theoretical rigor with practical implementation, 
particularly in structuring and democratizing access to scholarly knowledge. 
This PhD will equip me with the interdisciplinary expertise—spanning mathematics, computer science, 
and information science—to establish a research lab focused on advancing open knowledge infrastructures. 
TIB's emphasis on translational research and its global academic network will be instrumental in realizing this vision.

I am confident that my technical expertise in AI/ML, experience in managing complex data workflows, 
and alignment with TIB's strategic priorities position me to contribute meaningfully to your program. 
I would welcome the opportunity to discuss how my background could support initiatives like optimizing knowledge graph scalability or advancing explainability in AI-driven scholarly tools. 
Thank you for considering my application. 
I look forward to contributing to TIB's legacy of innovation in open science.

\makeletterclosing
%%%%%%%%%%%%%%%%%%%%%%%%%%%%%%%%%%%%%%%%%%%%%%%%%%%%%%%%%%%%%%%%%%%%%%%%%%%%%%%%%%%%%%%%%%%%%%%%%%%%%%%%%%%%%%%%%%%%%
\end{document}