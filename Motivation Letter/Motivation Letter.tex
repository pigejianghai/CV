\documentclass[11pt,a4paper, final]{moderncv}
\usepackage{color}
\usepackage{fontspec}
\usepackage{url}
\moderncvtheme{classic}
\definecolor{color1}{RGB}{1, 52, 64}
\usepackage[scale=0.88]{geometry}
\setlength{\hintscolumnwidth}{3.5cm} 
\AtBeginDocument{\recomputelengths} 
\usepackage{xunicode}
\usepackage{xltxtra}
\usepackage[utf8]{inputenc}
\usepackage{lipsum} 
\usepackage{tikz}
\usepackage{arydshln}
\usepackage{nccrules}
\usepackage{verbatim}
\setmainfont{Times New Roman}
\AfterPreamble{\hypersetup{
  pdfcreator={XeLaTeX},
  pdftitle={Fictional Cover Letter of Hai Jiang}
}}
\usepackage{fontawesome5}
%%%%%%%%%%%%%%%%%%%%%%%%%%%%%%%%%%%%%%%%%%%%%%%%%%%%%%%%%%%%%%%%%%%%%%%%%%%%%%%%%%%%%%%%%%%%%%%%%%%%%%%%%%%%%%%%%%%%%
%%%%%%%%%%%%%%%%%%%%%%%%%%%%%%%%%%%%%%%%%%%%%%%%%%%%%%%%%%%%%%%%%%%%%%%%%%%%%%%%%%%%%%%%%%%%%%%%%%%%%%%%%%%%%%%%%%%%%
%%%%%%%%%%%%%%%%%%%%%%%%%%%%%%%%%%%%%%%%%%%%%%%%%%%%%%%%%%%%%%%%%%%%%%%%%%%%%%%%%%%%%%%%%%%%%%%%%%%%%%%%%%%%%%%%%%%%%
%%%%%%%%%%%%%%%%%%%%%%%%%%%%%%%%%%%%%%%%%%%%%%%%%%%%%%%%%%%%%%%%%%%%%%%%%%%%%%%%%%%%%%%%%%%%%%%%%%%%%%%%%%%%%%%%%%%%%
%%%%%%%%%%%%%%%%%%%%%%%%%%%%%%%%%%%%%%%%%%%%%%%%%%%%%%%%%%%%%%%%%%%%%%%%%%%%%%%%%%%%%%%%%%%%%%%%%%%%%%%%%%%%%%%%%%%%%
\firstname{Hai}
\familyname{Jiang}
\address{Shenzhen, China}{}
\extrainfo{
	\faPhone\hspace{0.5em}+86 19867713757\\
	{\small\faEnvelope}\hspace{0.5em}jiangh14@lzu.edu.cn\\
}
\newcommand{\spacesection}{\vspace{0.4cm}}
\newcommand{\spacesubsection}{\vspace{0.2cm}}
%===========================
\begin{document}
\recipient{Admission Committee}
{Fraunhofer SCAI | University of Bonn\\
Artificial Intelligence in Parkinson's Disease (AIPD)\\
Marie Skłodowska-Curie Actions (MSCA)
% Marie Skłodowska-Curie Actions (MSCA)
% System State Monitoring (STM) \\
% Science and Technology\\
% Logistics and Supply Chain Management
}
\date{\today}
\opening{Dear Members of the Admissions Committee,}
\closing{Sincerely,}
%%%%%%%%%%%%%%%%%%%%%%%%%%%%%%%%%%%%%%%%%%%%%%%%%%%%%%%%%%%%%%%%%%%%%%%%%%%%%%%%%%%%%%%%%%%%%%%%%%%%%%%%%%%%%%%%%%%%%
%%%%%%%%%%%%%%%%%%%%%%%%%%%%%%%%%%%%%%%%%%%%%%%%%%%%%%%%%%%%%%%%%%%%%%%%%%%%%%%%%%%%%%%%%%%%%%%%%%%%%%%%%%%%%%%%%%%%%
%%%%%%%%%%%%%%%%%%%%%%%%%%%%%%%%%%%%%%%%%%%%%%%%%%%%%%%%%%%%%%%%%%%%%%%%%%%%%%%%%%%%%%%%%%%%%%%%%%%%%%%%%%%%%%%%%%%%%
%%%%%%%%%%%%%%%%%%%%%%%%%%%%%%%%%%%%%%%%%%%%%%%%%%%%%%%%%%%%%%%%%%%%%%%%%%%%%%%%%%%%%%%%%%%%%%%%%%%%%%%%%%%%%%%%%%%%%
%%%%%%%%%%%%%%%%%%%%%%%%%%%%%%%%%%%%%%%%%%%%%%%%%%%%%%%%%%%%%%%%%%%%%%%%%%%%%%%%%%%%%%%%%%%%%%%%%%%%%%%%%%%%%%%%%%%%%
\makelettertitle
\thispagestyle{empty}
\pagestyle{empty}
I am writing to express my enthusiastic interest in the PhD position 
in \textbf{``Generative AI for Simulating a Digital Parkinson's Disease (PD) Twin''} (AIPD-DC13-FH) 
at the University of Bonn. 
With a robust foundation in computational mathematics, 
hands-on experience in AI-driven medical data transformation, 
and a passion for advancing personalized healthcare through generative models, 
I am eager to contribute to MSCA's vision of encoding multimodal longitudinal patient data into latent ODE systems for predictive disease modeling.

% \textbf{Academic Background \& Research Expertise}\\
During my Master's in Computational Mathematics (2019-2022), 
I specialized in \textbf{Generative Adversarial Networks (GANs)}, 
focusing on cross-domain style transfer for structured data. 
My thesis, \emph{``GANs for Personal Style Imitation of Chinese Handwritten Characters,''} 
nvolved designing an end-to-end CycleGAN framework to replicate the nuanced calligraphic style of Shiling Shen Chern. 
By optimizing adversarial training and domain-specific preprocessing, 
I achieved \textbf{85\% visual similarity} across 220 characters, outperforming baseline models by 10\%. 
This project honed my ability to map heterogeneous data domains—a skill directly applicable to simulating personalized PD trajectories. 
For example, adversarial training principles could be adapted to generate synthetic patient data under clinical constraints, enabling robust modeling of disease progression.

As a researcher at Sun Yat-sen University's \textbf{Computational Medical Imaging Lab}, 
I led interdisciplinary projects that combined AI innovation with clinical relevance. 
For example, I developed a \textbf{multi-task learning model} to 
classify Placenta Accreta Spectrum Disorder severity using T2-WI MRI images, achieving an \textbf{AUC of 0.80}. 
In a separate project, 
I designed a CNN-based system to predict breast cancer metastasis in Sentinel Lymph Nodes via dual-energy CT scans, 
attaining an \textbf{AUC of 0.85} in cross-validation. 
Beyond algorithm development, I curated and harmonized heterogeneous DICOM datasets while deploying models on hospital servers, 
demonstrating my proficiency in managing complex medical data workflows—a critical competency for integrating multimodal PD data (e.g., imaging, genomics, and sensor inputs) into cohesive digital twins. 
These efforts resulted in two first-authored papers (one accepted at ISBI 2025, one under revision).

% \textbf{Alignment}\\
Your program's emphasis on \textbf{generative AI approach for “personalized” simulation of disease outcome trajectories} 
aligns with my aspiration to develop interpretable, patient-specific models for disease trajectory prediction. 
My expertise in GANs for structured data transformation directly supports the challenge of synthesizing longitudinal patient data under clinical constraints. 
For example, domain adaptation strategies from my thesis could refine latent ODE systems to capture individual disease dynamics, while my CNN-based medical imaging work could enhance feature extraction from brain scans or wearable sensor data. 
Additionally, my experience in deploying validated clinical models underscores my ability to translate theoretical AI advancements into tools that address real-world PD challenges.

A PhD at the University of Bonn would provide the ideal interdisciplinary environment—spanning AI, computational biology, and neurology—to advance my goal of pioneering explainable generative models for precision medicine. 
Long-term, I aim to lead research focused on integrating multimodal data (e.g., DaTSCAN imaging, motor symptom logs, and genetic profiles) to create dynamic digital twins that predict patient-specific responses to therapies. 
The University's translational ethos, state-of-the-art facilities, and partnerships with NCER-PD/LuxPark and PPMI would empower me to validate these models in clinical cohorts, aligning with Fraunhofer SCAI's legacy in AI-driven healthcare innovation.

% \textbf{Long-Term Goals}\\
The MSCA program's focus on AI for neurodegenerative diseases offers unparalleled opportunities to refine my expertise in generative modeling and dynamical systems. 
I am particularly excited to collaborate on projects such as physics-informed neural networks for latent ODE optimization or federated learning frameworks to ensure privacy-compliant PD data sharing. 
My technical background in adversarial training, coupled with my commitment to interdisciplinary problem-solving, positions me to contribute meaningfully to the AIPD-DC13-FH project and advance Bonn's strategic objectives in digital health.

Thank you for considering my application. 
I would welcome the opportunity to discuss how my expertise in generative AI, medical data harmonization, and cross-domain adaptation could support MSCA's mission. 
I look forward to contributing to groundbreaking research at the intersection of AI, computational medicine, and personalized neurology.

\makeletterclosing
%%%%%%%%%%%%%%%%%%%%%%%%%%%%%%%%%%%%%%%%%%%%%%%%%%%%%%%%%%%%%%%%%%%%%%%%%%%%%%%%%%%%%%%%%%%%%%%%%%%%%%%%%%%%%%%%%%%%%
\end{document}