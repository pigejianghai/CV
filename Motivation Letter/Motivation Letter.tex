\documentclass[11pt,a4paper, final]{moderncv}
\usepackage{color}
\usepackage{fontspec}
\usepackage{url}
\moderncvtheme{classic}
\definecolor{color1}{RGB}{1, 52, 64}
\usepackage[scale=0.92]{geometry}
\setlength{\hintscolumnwidth}{3.5cm} 
\AtBeginDocument{\recomputelengths} 
\usepackage{xunicode}
\usepackage{xltxtra}
\usepackage[utf8]{inputenc}
\usepackage{lipsum} 
\usepackage{tikz}
\usepackage{arydshln}
\usepackage{nccrules}
\usepackage{verbatim}
\setmainfont{Times New Roman}
\AfterPreamble{\hypersetup{
  pdfcreator={XeLaTeX},
  pdftitle={Fictional Cover Letter of Hai Jiang}
}}
\usepackage{fontawesome5}
%%%%%%%%%%%%%%%%%%%%%%%%%%%%%%%%%%%%%%%%%%%%%%%%%%%%%%%%%%%%%%%%%%%%%%%%%%%%%%%%%%%%%%%%%%%%%%%%%%%%%%%%%%%%%%%%%%%%%
%%%%%%%%%%%%%%%%%%%%%%%%%%%%%%%%%%%%%%%%%%%%%%%%%%%%%%%%%%%%%%%%%%%%%%%%%%%%%%%%%%%%%%%%%%%%%%%%%%%%%%%%%%%%%%%%%%%%%
%%%%%%%%%%%%%%%%%%%%%%%%%%%%%%%%%%%%%%%%%%%%%%%%%%%%%%%%%%%%%%%%%%%%%%%%%%%%%%%%%%%%%%%%%%%%%%%%%%%%%%%%%%%%%%%%%%%%%
%%%%%%%%%%%%%%%%%%%%%%%%%%%%%%%%%%%%%%%%%%%%%%%%%%%%%%%%%%%%%%%%%%%%%%%%%%%%%%%%%%%%%%%%%%%%%%%%%%%%%%%%%%%%%%%%%%%%%
%%%%%%%%%%%%%%%%%%%%%%%%%%%%%%%%%%%%%%%%%%%%%%%%%%%%%%%%%%%%%%%%%%%%%%%%%%%%%%%%%%%%%%%%%%%%%%%%%%%%%%%%%%%%%%%%%%%%%
\firstname{Hai}
\familyname{Jiang}
\address{Shenzhen, China}{}
\extrainfo{
	\faPhone\hspace{0.5em}+86 19867713757\\
	{\small\faEnvelope}\hspace{0.5em}jiangh14@lzu.edu.cn\\
}
\newcommand{\spacesection}{\vspace{0.4cm}}
\newcommand{\spacesubsection}{\vspace{0.2cm}}
%===========================
\begin{document}
\recipient{Admission Committee}
{University of Athens\\
Department of Informatics and Telecommunications | BICEPS\\
Marie Skłodowska-Curie Actions (MSCA)
}
\date{\today}
\opening{Dear Members of the Admissions Committee,}
\closing{Sincerely,}
%%%%%%%%%%%%%%%%%%%%%%%%%%%%%%%%%%%%%%%%%%%%%%%%%%%%%%%%%%%%%%%%%%%%%%%%%%%%%%%%%%%%%%%%%%%%%%%%%%%%%%%%%%%%%%%%%%%%%
%%%%%%%%%%%%%%%%%%%%%%%%%%%%%%%%%%%%%%%%%%%%%%%%%%%%%%%%%%%%%%%%%%%%%%%%%%%%%%%%%%%%%%%%%%%%%%%%%%%%%%%%%%%%%%%%%%%%%
%%%%%%%%%%%%%%%%%%%%%%%%%%%%%%%%%%%%%%%%%%%%%%%%%%%%%%%%%%%%%%%%%%%%%%%%%%%%%%%%%%%%%%%%%%%%%%%%%%%%%%%%%%%%%%%%%%%%%
%%%%%%%%%%%%%%%%%%%%%%%%%%%%%%%%%%%%%%%%%%%%%%%%%%%%%%%%%%%%%%%%%%%%%%%%%%%%%%%%%%%%%%%%%%%%%%%%%%%%%%%%%%%%%%%%%%%%%
%%%%%%%%%%%%%%%%%%%%%%%%%%%%%%%%%%%%%%%%%%%%%%%%%%%%%%%%%%%%%%%%%%%%%%%%%%%%%%%%%%%%%%%%%%%%%%%%%%%%%%%%%%%%%%%%%%%%%
\makelettertitle
\thispagestyle{empty}
\pagestyle{empty}
I am writing to express my enthusiastic interest in the PhD position 
in \textbf{``AI pipeline development for metabolomics data analysis ''} at the University of Athens. 
With a Master's degree in Computational Mathematics, 
hands-on expertise in AI-driven medical data transformation, 
and a passion for advancing precision medicine through explainable AI, 
I am eager to contribute to BICEPS and MSCA's mission of investigating inflammation-driven mechanisms in Parkinson's Disease (PD).

% \textbf{Academic Background \& Research Expertise}\\
During my Master's in Computational Mathematics (2019-2022), 
I specialized in \textbf{Generative Adversarial Networks (GANs)}, 
focusing on cross-domain style transfer for structured data. 
My thesis, \emph{``GANs for Personal Style Imitation of Chinese Handwritten Characters,''} 
nvolved designing an end-to-end CycleGAN framework to replicate the nuanced calligraphic style of Shiling Shen Chern. 
By optimizing adversarial training and domain-specific preprocessing, 
I achieved \textbf{85\% visual similarity} across 220 characters, outperforming baseline models by 10\%. 
This work demonstrated my ability to adapt generative models for personalized data—a skill directly applicable 
to integrating multi-omics data (e.g., metabolomics, genomics, and imaging) for PD biomarker discovery.

As a researcher at Sun Yat-sen University's \textbf{Computational Medical Imaging Lab}, 
I led interdisciplinary projects that combined AI innovation with clinical relevance. 
For example, I developed a \textbf{multi-task learning model} to 
classify Placenta Accreta Spectrum Disorder severity using T2-WI MRI images, achieving an \textbf{AUC of 0.80}. 
In a separate project, 
I designed a CNN-based system to predict breast cancer metastasis in Sentinel Lymph Nodes via dual-energy CT scans, 
attaining an \textbf{AUC of 0.85} in cross-validation. 
These experiences equipped me with expertise in building robust AI pipelines for heterogeneous medical data—critical for addressing PD's complexity. 
For example, my work on domain adaptation could align metabolomic profiles with clinical phenotypes, while my CNN expertise could extract inflammation-related biomarkers from brain imaging.

% \textbf{Alignment}\\
Your program's focus on \textbf{explainable AI for metabolomics data analysis} aligns with my aspiration 
to develop interpretable, multi-modal frameworks for PD. 
During my PhD, I propose to: 
1. Design graph neural networks (GNNs) to model PD progression by integrating metabolomic, proteomic, and imaging data; 
2. Develop vision transformer (ViT)-based pipelines to identify inflammation-driven metabolic signatures in longitudinal cohorts; 
3. Validate these tools in collaboration with clinical partners to uncover novel therapeutic targets.

My long-term goal is to pioneer AI-driven workflows that translate complex metabolomic data into actionable insights for precision medicine. 
The integration of GNNs and ViTs could enable dynamic modeling of neuroinflammatory pathways, directly addressing BICEPS's aim to unravel PD's molecular mechanisms.

% \textbf{Long-Term Goals}\\
The MSCA program's emphasis on AI for PD and Athens's access to longitudinal PD cohorts provide an unparalleled environment to advance my research. 
I am particularly inspired by BICEPS's interdisciplinary ethos and partnerships with industry leaders, 
which align with my vision of bridging computational innovation and clinical practice. 
The opportunity to collaborate with industry further ensures methodological rigor and translational impact.

I am eager to contribute my expertise in AI/ML, data harmonization, and cross-domain adaptation to BICEPS's strategic goals. 
Thank you for considering my application. 
I would welcome the opportunity to discuss how my background aligns with your research priorities. 
Together, we can drive groundbreaking advancements at the intersection of AI, metabolomics, and PD therapeutics.

\makeletterclosing
%%%%%%%%%%%%%%%%%%%%%%%%%%%%%%%%%%%%%%%%%%%%%%%%%%%%%%%%%%%%%%%%%%%%%%%%%%%%%%%%%%%%%%%%%%%%%%%%%%%%%%%%%%%%%%%%%%%%%
\end{document}