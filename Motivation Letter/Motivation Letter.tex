\documentclass[11pt,a4paper, final]{moderncv}
\usepackage{color}
\usepackage{fontspec}
\usepackage{url}
\moderncvtheme{classic}
\definecolor{color1}{RGB}{1, 52, 64}
\usepackage[scale=0.88]{geometry}
\setlength{\hintscolumnwidth}{3.5cm} 
\AtBeginDocument{\recomputelengths} 
\usepackage{xunicode}
\usepackage{xltxtra}
\usepackage[utf8]{inputenc}
\usepackage{lipsum} 
\usepackage{tikz}
\usepackage{arydshln}
\usepackage{nccrules}
\usepackage{verbatim}
\setmainfont{Times New Roman}
\AfterPreamble{\hypersetup{
  pdfcreator={XeLaTeX},
  pdftitle={Fictional Cover Letter of Hai Jiang}
}}
\usepackage{fontawesome5}
%%%%%%%%%%%%%%%%%%%%%%%%%%%%%%%%%%%%%%%%%%%%%%%%%%%%%%%%%%%%%%%%%%%%%%%%%%%%%%%%%%%%%%%%%%%%%%%%%%%%%%%%%%%%%%%%%%%%%
%%%%%%%%%%%%%%%%%%%%%%%%%%%%%%%%%%%%%%%%%%%%%%%%%%%%%%%%%%%%%%%%%%%%%%%%%%%%%%%%%%%%%%%%%%%%%%%%%%%%%%%%%%%%%%%%%%%%%
%%%%%%%%%%%%%%%%%%%%%%%%%%%%%%%%%%%%%%%%%%%%%%%%%%%%%%%%%%%%%%%%%%%%%%%%%%%%%%%%%%%%%%%%%%%%%%%%%%%%%%%%%%%%%%%%%%%%%
%%%%%%%%%%%%%%%%%%%%%%%%%%%%%%%%%%%%%%%%%%%%%%%%%%%%%%%%%%%%%%%%%%%%%%%%%%%%%%%%%%%%%%%%%%%%%%%%%%%%%%%%%%%%%%%%%%%%%
%%%%%%%%%%%%%%%%%%%%%%%%%%%%%%%%%%%%%%%%%%%%%%%%%%%%%%%%%%%%%%%%%%%%%%%%%%%%%%%%%%%%%%%%%%%%%%%%%%%%%%%%%%%%%%%%%%%%%
\firstname{Hai}
\familyname{Jiang}
\address{Shenzhen, China}{}
\extrainfo{
	\faPhone\hspace{0.5em}+86 19867713757\\
	{\small\faEnvelope}\hspace{0.5em}jiangh14@lzu.edu.cn\\
}
\newcommand{\spacesection}{\vspace{0.4cm}}
\newcommand{\spacesubsection}{\vspace{0.2cm}}
%===========================
\begin{document}
\recipient{Admission Committee}
{Bayer A.G. Germany\\
Marie Skłodowska-Curie Actions (MSCA)
}
\date{\today}
\opening{Dear Members of the Admissions Committee,}
\closing{Sincerely,}
%%%%%%%%%%%%%%%%%%%%%%%%%%%%%%%%%%%%%%%%%%%%%%%%%%%%%%%%%%%%%%%%%%%%%%%%%%%%%%%%%%%%%%%%%%%%%%%%%%%%%%%%%%%%%%%%%%%%%
%%%%%%%%%%%%%%%%%%%%%%%%%%%%%%%%%%%%%%%%%%%%%%%%%%%%%%%%%%%%%%%%%%%%%%%%%%%%%%%%%%%%%%%%%%%%%%%%%%%%%%%%%%%%%%%%%%%%%
%%%%%%%%%%%%%%%%%%%%%%%%%%%%%%%%%%%%%%%%%%%%%%%%%%%%%%%%%%%%%%%%%%%%%%%%%%%%%%%%%%%%%%%%%%%%%%%%%%%%%%%%%%%%%%%%%%%%%
%%%%%%%%%%%%%%%%%%%%%%%%%%%%%%%%%%%%%%%%%%%%%%%%%%%%%%%%%%%%%%%%%%%%%%%%%%%%%%%%%%%%%%%%%%%%%%%%%%%%%%%%%%%%%%%%%%%%%
%%%%%%%%%%%%%%%%%%%%%%%%%%%%%%%%%%%%%%%%%%%%%%%%%%%%%%%%%%%%%%%%%%%%%%%%%%%%%%%%%%%%%%%%%%%%%%%%%%%%%%%%%%%%%%%%%%%%%
\makelettertitle
\thispagestyle{empty}
\pagestyle{empty}
I am writing to express my enthusiastic interest in the PhD position 
in \textbf{``AI-driven data analysis, target identification and treatment development in PD''}. 
With a Master's degree in Computational Mathematics, hands-on expertise in AI-driven medical imaging, 
and a passion for translating algorithmic innovations into clinical solutions, 
I am eager to contribute to Bayer and MSCA's mission of unraveling neuro-immune mechanisms in PD through cutting-edge AI.

% \textbf{Academic Background \& Research Expertise}\\
During my Master's in Computational Mathematics (2019-2022), 
I specialized in \textbf{Generative Adversarial Networks (GANs)}, 
focusing on cross-domain style transfer for structured data. 
My thesis, \emph{``GANs for Personal Style Imitation of Chinese Handwritten Characters,''} 
nvolved designing an end-to-end CycleGAN framework to replicate the nuanced calligraphic style of Shiling Shen Chern. 
By optimizing adversarial training and domain-specific preprocessing, 
I achieved \textbf{85\% visual similarity} across 220 characters, outperforming baseline models by 10\%. 
This work demonstrated my ability to adapt generative models for structured data—a skill directly applicable to integrating multimodal PD datasets (e.g., MRI, genomics, and clinical records).

As a researcher at Sun Yat-sen University's \textbf{Computational Medical Imaging Lab}, 
I led interdisciplinary projects that combined AI innovation with clinical relevance. 
For example, I developed a \textbf{multi-task learning model} to 
classify Placenta Accreta Spectrum Disorder severity using T2-WI MRI images, achieving an \textbf{AUC of 0.80}. 
In a separate project, 
I designed a CNN-based system to predict breast cancer metastasis in Sentinel Lymph Nodes via dual-energy CT scans, 
attaining an \textbf{AUC of 0.85} in cross-validation. 
These experiences honed my ability to design robust AI pipelines for heterogeneous medical data—skills critical for addressing PD's complexity. 
For instance, my work on domain adaptation could harmonize disparate PD datasets, while my expertise in CNNs aligns with extracting biomarkers from brain imaging.

% \textbf{Alignment}\\
Your program's focus on graph-based knowledge models and personalized treatment strategies resonates with my goal to develop interpretable AI frameworks for PD. 
During my PhD, I propose to: 
1. Design graph neural networks (GNNs) to map PD progression by integrating multimodal data (e.g., MRI, proteomics, and patient diaries); 
2. Collaborate with FAIR's computational teams to validate these tools in clinical trials, ensuring translational impact.

A PhD at Bayer A.G. Germany would provide the ideal interdisciplinary environment—spanning AI, biology, and medicine—to advance my goal of pioneering explainable GNNs models for precision medicine. 
Long-term, I aim to lead research focused on integrating multimodal data (e.g., GNNs, and GCNs) to create dynamic digital twins that predict patient-specific responses to therapies. 
The group's translational ethos, state-of-the-art facilities, and partnerships with FAIR would empower me to validate these models in clinical cohorts, aligning with BICEPS's legacy in AI-driven healthcare innovation. 
This vision aligns with MSCA's aim to build AI-driven “hypothesis generation” systems and Bayer's translational research ethos.

% \textbf{Long-Term Goals}\\
I aspire to lead AI-driven initiatives that bridge computational innovation and clinical practice, ultimately advancing precision medicine for neurodegenerative diseases. 
Bayer's state-of-the-art facilities, partnerships with FAIR, and access to longitudinal PD cohorts (e.g., the PPMI dataset) provide an unparalleled environment to realize this ambition.

Thank you for considering my application. 
I would welcome the opportunity to discuss how my background in AI/ML, medical data harmonization, and cross-domain adaptation aligns with your research priorities. 
I am eager to contribute to groundbreaking work at the intersection of AI, neuroimmunology, and PD therapeutics.

\makeletterclosing
%%%%%%%%%%%%%%%%%%%%%%%%%%%%%%%%%%%%%%%%%%%%%%%%%%%%%%%%%%%%%%%%%%%%%%%%%%%%%%%%%%%%%%%%%%%%%%%%%%%%%%%%%%%%%%%%%%%%%
\end{document}