\documentclass[11pt,a4paper, final]{moderncv}
\usepackage{color}
\usepackage{fontspec}
\usepackage{url}
\moderncvtheme{classic}
\definecolor{color1}{RGB}{1, 52, 64}
\usepackage[scale=0.92]{geometry}
\setlength{\hintscolumnwidth}{3.5cm} 
\AtBeginDocument{\recomputelengths} 
\usepackage{xunicode}
\usepackage{xltxtra}
\usepackage[utf8]{inputenc}
\usepackage{lipsum} 
\usepackage{tikz}
\usepackage{arydshln}
\usepackage{nccrules}
\usepackage{verbatim}
\setmainfont{Times New Roman}
\AfterPreamble{\hypersetup{
  pdfcreator={XeLaTeX},
  pdftitle={Fictional Cover Letter of Hai Jiang}
}}
\usepackage{fontawesome5}
%%%%%%%%%%%%%%%%%%%%%%%%%%%%%%%%%%%%%%%%%%%%%%%%%%%%%%%%%%%%%%%%%%%%%%%%%%%%%%%%%%%%%%%%%%%%%%%%%%%%%%%%%%%%%%%%%%%%%
%%%%%%%%%%%%%%%%%%%%%%%%%%%%%%%%%%%%%%%%%%%%%%%%%%%%%%%%%%%%%%%%%%%%%%%%%%%%%%%%%%%%%%%%%%%%%%%%%%%%%%%%%%%%%%%%%%%%%
%%%%%%%%%%%%%%%%%%%%%%%%%%%%%%%%%%%%%%%%%%%%%%%%%%%%%%%%%%%%%%%%%%%%%%%%%%%%%%%%%%%%%%%%%%%%%%%%%%%%%%%%%%%%%%%%%%%%%
%%%%%%%%%%%%%%%%%%%%%%%%%%%%%%%%%%%%%%%%%%%%%%%%%%%%%%%%%%%%%%%%%%%%%%%%%%%%%%%%%%%%%%%%%%%%%%%%%%%%%%%%%%%%%%%%%%%%%
%%%%%%%%%%%%%%%%%%%%%%%%%%%%%%%%%%%%%%%%%%%%%%%%%%%%%%%%%%%%%%%%%%%%%%%%%%%%%%%%%%%%%%%%%%%%%%%%%%%%%%%%%%%%%%%%%%%%%
\firstname{Hai}
\familyname{Jiang}
\address{Shenzhen, China}{}
\extrainfo{
	\faPhone\hspace{0.5em}+86 19867713757\\
	{\small\faEnvelope}\hspace{0.5em}jiangh14@lzu.edu.cn\\
}
\newcommand{\spacesection}{\vspace{0.4cm}}
\newcommand{\spacesubsection}{\vspace{0.2cm}}
%===========================
\begin{document}
\recipient{Admission Committee}
{Erasmus Medical Center (EMC)\\
Artificial Intelligence in Parkinson's Disease (AIPD)\\
Marie Skłodowska-Curie Actions (MSCA)
}
\date{\today}
\opening{Dear Members of the Admissions Committee,}
\closing{Sincerely,}
%%%%%%%%%%%%%%%%%%%%%%%%%%%%%%%%%%%%%%%%%%%%%%%%%%%%%%%%%%%%%%%%%%%%%%%%%%%%%%%%%%%%%%%%%%%%%%%%%%%%%%%%%%%%%%%%%%%%%
%%%%%%%%%%%%%%%%%%%%%%%%%%%%%%%%%%%%%%%%%%%%%%%%%%%%%%%%%%%%%%%%%%%%%%%%%%%%%%%%%%%%%%%%%%%%%%%%%%%%%%%%%%%%%%%%%%%%%
%%%%%%%%%%%%%%%%%%%%%%%%%%%%%%%%%%%%%%%%%%%%%%%%%%%%%%%%%%%%%%%%%%%%%%%%%%%%%%%%%%%%%%%%%%%%%%%%%%%%%%%%%%%%%%%%%%%%%
%%%%%%%%%%%%%%%%%%%%%%%%%%%%%%%%%%%%%%%%%%%%%%%%%%%%%%%%%%%%%%%%%%%%%%%%%%%%%%%%%%%%%%%%%%%%%%%%%%%%%%%%%%%%%%%%%%%%%
%%%%%%%%%%%%%%%%%%%%%%%%%%%%%%%%%%%%%%%%%%%%%%%%%%%%%%%%%%%%%%%%%%%%%%%%%%%%%%%%%%%%%%%%%%%%%%%%%%%%%%%%%%%%%%%%%%%%%
\makelettertitle
\thispagestyle{empty}
\pagestyle{empty}
I am writing to express my enthusiastic interest in the PhD position 
\textbf{Early-Stage Diagnosis of Parkinson and Atypical Parkinsonism using Quantitative MRI Biomarkers and Artificial Intelligence} (AIPD-DC10-EMC) at the Erasmus Medical Center (EMC). 
With a Master's degree in Computational Mathematics, 
hands-on expertise in AI-driven medical image analysis, 
and a passion for advancing precision neurology, 
I am eager to contribute to AIPD's mission of leveraging quantitative MRI and AI to transform early diagnosis and understanding of Parkinson's disease (PD).

% \textbf{Academic Background \& Research Expertise}\\
During my Master's in Computational Mathematics (2019-2022), 
I specialized in \textbf{Generative Adversarial Networks (GANs)}, 
focusing on cross-domain style transfer for structured data. 
My thesis, \emph{``GANs for Personal Style Imitation of Chinese Handwritten Characters,''} 
nvolved designing an end-to-end CycleGAN framework to replicate the nuanced calligraphic style of Shiling Shen Chern. 
By optimizing adversarial training and domain-specific preprocessing, 
I achieved \textbf{85\% visual similarity} across 220 characters, outperforming baseline models by 10\%. 
This project honed my ability to harmonize heterogeneous data domains—a skill directly applicable to integrating multimodal MRI biomarkers for PD subtype differentiation.

As a researcher at Sun Yat-sen University's Computational Medical Imaging Lab, 
I led interdisciplinary projects that combined AI innovation with clinical impact. 
For example: 
(1) Developed a \textbf{multi-task learning model} to classify Placenta Accreta Spectrum Disorder severity using T2-weighted MRI images, achieving an \textbf{AUC of 0.80}; 
(2) Designed a \textbf{CNN-based system} to predict breast cancer metastasis in Sentinel Lymph Nodes via dual-energy CT scans, attaining an \textbf{AUC of 0.85} in cross-validation.
These experiences equipped me with expertise in building robust AI pipelines for high-dimensional medical data—critical for addressing challenges in PD, 
such as distinguishing early-stage Parkinsonism from atypical variants. 
My work on adversarial training and domain adaptation could enhance the generalizability of AI models across diverse MRI datasets, 
while my CNN expertise aligns with extracting subtle biomarkers from structural and functional imaging.

% \textbf{Alignment}\\
Your program's focus on quantitative MRI biomarkers and AI for early PD diagnosis resonates with my aspiration to pioneer interpretable, data-driven frameworks for neurodegenerative diseases. 
During my PhD, I propose to: 
(1) Develop \textbf{GAN-based synthetic data pipelines} to augment limited MRI datasets, improving model robustness for rare PD subtypes; 
(2) Design \textbf{hybrid CNN-GNN architectures} to integrate MRI biomarkers with clinical data, enabling personalized disease trajectory prediction; 
(3) \textbf{Validate} these tools in multi-center cohorts to identify early diagnostic signatures of atypical Parkinsonism.

% \textbf{Long-Term Goals}\\
A PhD at EMC would provide the ideal environment to advance my goal of creating an AI “one-stop-shop” diagnostic tool for early PD detection. 
Long-term, I aim to lead research on integrating explainable AI with quantitative MRI to uncover neuroinflammatory pathways and accelerate therapeutic discovery. 
AIPD's partnerships with leading hospitals and access to longitudinal PD cohorts align perfectly with my vision of translating computational innovations into clinical practice.

EMC's world-class infrastructure, expertise in quantitative MRI, and commitment to AI-driven neurology offer unparalleled opportunities to refine my skills. 
I am particularly inspired by AIPD's interdisciplinary approach and collaborations with institutions like the MSCA network, 
which will enable me to bridge cutting-edge AI methodologies with biomarker validation in real-world settings.

I am eager to contribute my expertise in adversarial learning, medical imaging, and interpretable AI to AIPD's strategic goals. 
Thank you for considering my application. 
I would welcome the opportunity to discuss how my background aligns with your research priorities. 
Together, we can drive advancements at the intersection of AI, neuroimaging, and precision medicine for Parkinson's disease.

\makeletterclosing
%%%%%%%%%%%%%%%%%%%%%%%%%%%%%%%%%%%%%%%%%%%%%%%%%%%%%%%%%%%%%%%%%%%%%%%%%%%%%%%%%%%%%%%%%%%%%%%%%%%%%%%%%%%%%%%%%%%%%
\end{document}