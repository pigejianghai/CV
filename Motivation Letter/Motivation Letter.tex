\documentclass[11pt,a4paper, final]{moderncv}
\usepackage{color}
\usepackage{fontspec}
\usepackage{url}
\moderncvtheme{classic}
\definecolor{color1}{RGB}{1, 52, 64}
\usepackage[scale=0.95]{geometry}
\setlength{\hintscolumnwidth}{3.5cm} 
\AtBeginDocument{\recomputelengths} 
\usepackage{xunicode}
\usepackage{xltxtra}
\usepackage[utf8]{inputenc}
\usepackage{lipsum} 
\usepackage{tikz}
\usepackage{arydshln}
\usepackage{nccrules}
\usepackage{verbatim}
\setmainfont{Times New Roman}
\AfterPreamble{\hypersetup{
  pdfcreator={XeLaTeX},
  pdftitle={Fictional CV of Hai Jiang}
}}
\usepackage{fontawesome5}
%%%%%%%%%%%%%%%%%%%%%%%%%%%%%%%%%%%%%%%%%%%%%%%%%%%%%%%%%%%%%%%%%%%%%%%%%%%%%%%%%%%%%%%%%%%%%%%%%%%%%%%%%%%%%%%%%%%%%
%%%%%%%%%%%%%%%%%%%%%%%%%%%%%%%%%%%%%%%%%%%%%%%%%%%%%%%%%%%%%%%%%%%%%%%%%%%%%%%%%%%%%%%%%%%%%%%%%%%%%%%%%%%%%%%%%%%%%
%%%%%%%%%%%%%%%%%%%%%%%%%%%%%%%%%%%%%%%%%%%%%%%%%%%%%%%%%%%%%%%%%%%%%%%%%%%%%%%%%%%%%%%%%%%%%%%%%%%%%%%%%%%%%%%%%%%%%
%%%%%%%%%%%%%%%%%%%%%%%%%%%%%%%%%%%%%%%%%%%%%%%%%%%%%%%%%%%%%%%%%%%%%%%%%%%%%%%%%%%%%%%%%%%%%%%%%%%%%%%%%%%%%%%%%%%%%
%%%%%%%%%%%%%%%%%%%%%%%%%%%%%%%%%%%%%%%%%%%%%%%%%%%%%%%%%%%%%%%%%%%%%%%%%%%%%%%%%%%%%%%%%%%%%%%%%%%%%%%%%%%%%%%%%%%%%
\firstname{Hai}
\familyname{Jiang}
% \address{Shenzhen, China}{}
\extrainfo{
	\faPhone\hspace{0.5em}+86 19867713757\\
	{\small\faEnvelope}\hspace{0.5em}jiangh14@lzu.edu.cn\\
}
\newcommand{\spacesection}{\vspace{0.4cm}}
\newcommand{\spacesubsection}{\vspace{0.2cm}}
%===========================
\begin{document}
\recipient{Admission Committee}
{FIT for High-Speed Dynamics, Ernst-Mach-Institut EMI\\
% Department of Resilience, Safety and Security\\
Marie Skłodowska-Curie Actions (MSCAs)\\
% System State Monitoring (STM) \\
% Science and Technology\\
% Logistics and Supply Chain Management
}
\date{\today}
\opening{Dear Members of the Admissions Committee,}
\closing{Sincerely,}
%%%%%%%%%%%%%%%%%%%%%%%%%%%%%%%%%%%%%%%%%%%%%%%%%%%%%%%%%%%%%%%%%%%%%%%%%%%%%%%%%%%%%%%%%%%%%%%%%%%%%%%%%%%%%%%%%%%%%
%%%%%%%%%%%%%%%%%%%%%%%%%%%%%%%%%%%%%%%%%%%%%%%%%%%%%%%%%%%%%%%%%%%%%%%%%%%%%%%%%%%%%%%%%%%%%%%%%%%%%%%%%%%%%%%%%%%%%
%%%%%%%%%%%%%%%%%%%%%%%%%%%%%%%%%%%%%%%%%%%%%%%%%%%%%%%%%%%%%%%%%%%%%%%%%%%%%%%%%%%%%%%%%%%%%%%%%%%%%%%%%%%%%%%%%%%%%
%%%%%%%%%%%%%%%%%%%%%%%%%%%%%%%%%%%%%%%%%%%%%%%%%%%%%%%%%%%%%%%%%%%%%%%%%%%%%%%%%%%%%%%%%%%%%%%%%%%%%%%%%%%%%%%%%%%%%
%%%%%%%%%%%%%%%%%%%%%%%%%%%%%%%%%%%%%%%%%%%%%%%%%%%%%%%%%%%%%%%%%%%%%%%%%%%%%%%%%%%%%%%%%%%%%%%%%%%%%%%%%%%%%%%%%%%%%
\makelettertitle
\thispagestyle{empty}
\pagestyle{empty}
I am writing to express my enthusiastic interest in the PhD position 
in \textbf{Automated Hidden Markov Modelling (HMM) for power network components} at the Fraunhofer Institute. 
With a robust foundation in computational mathematics, 
hands-on experience in AI-driven structured data transformation, 
and a passion for developing explainable models to enhance system resilience, 
I am eager to contribute to your mission of advancing autonomous inspection and health monitoring technologies for critical infrastructure.

% \textbf{Academic Background \& Research Expertise}\\
During my Master's in Computational Mathematics (2019-2022), 
I specialized in \textbf{Generative Adversarial Networks (GANs)}, 
focusing on cross-domain style transfer for structured data. 
My thesis, \emph{``GANs for Personal Style Imitation of Chinese Handwritten Characters,''} 
nvolved designing an end-to-end CycleGAN framework to replicate the nuanced calligraphic style of Shiling Shen Chern. 
By optimizing adversarial training and domain-specific preprocessing, 
I achieved \textbf{85\% visual similarity} across 220 characters, outperforming baseline models by 10\%. 
This project honed my ability to map heterogeneous data domains—a skill directly transferable to integrating AI into autonomous inspection systems, 
such as synthesizing HMM training data for power network components under varying operational conditions.

Post-graduation, I joined Sun Yat-sen University's \textbf{Computational Medical Imaging Lab}, 
where I led AI-driven projects with clinical impact. 
For instance, I developed a \textbf{multi-task learning model} to 
classify Placenta Accreta Spectrum Disorder severity using T2-WI MRI images, achieving an \textbf{AUC of 0.80}. 
In a separate project, 
I designed a CNN-based system to predict breast cancer metastasis in Sentinel Lymph Nodes via dual-energy CT scans, 
attaining an \textbf{AUC of 0.85} in cross-validation. 
Beyond algorithm development, I curated heterogeneous DICOM datasets and deployed models on hospital servers, demonstrating my proficiency in managing complex data pipelines—a critical competency for addressing uncertainties in power network data and ensuring robust HMM training.  
These efforts resulted in two first-authored papers (one accepted at ISBI 2025, one under revision).

% \textbf{Alignment}\\
Your program's emphasis on \textbf{automated HMM for system state modeling} 
aligns with my aspiration to develop interpretable, data-driven frameworks for infrastructure resilience. 
My experience with GANs for structured data transformation and CNNs for medical imaging directly supports the challenge of generating synthetic training data for HMMs in local electricity grids. 
For instance, adversarial training techniques I employed to preserve style fidelity in handwritten characters could be adapted to simulate diverse failure scenarios in power networks, enabling HMMs to better predict system states under uncertainty. 
Additionally, my work on deploying clinically validated models underscores my ability to bridge theoretical AI advancements with real-world engineering applications—key to ensuring HMMs are both accurate and actionable for stakeholders.

A PhD at Fraunhofer EMI would provide the ideal interdisciplinary environment—spanning AI, mathematics, and civil engineering—to advance my goal of pioneering Automated HMM Solutions that enhance infrastructure safety and sustainability. 
Long-term, I aim to establish a research group focused on AI-driven health monitoring systems, addressing challenges such as scalable training data generation, model interpretability, and real-time anomaly detection in power networks. 
The Institute's translational ethos and partnerships with industry leaders would empower me to drive innovations from theory to practice, aligning with Fraunhofer's legacy in resilience engineering.

% \textbf{Long-Term Goals}\\
The MSCA program's focus on AI and mathematical modeling offers unparalleled opportunities to refine my expertise in computational method development. 
I am particularly excited to collaborate on projects such as HMM-based anomaly detection in smart grids or integrating physics-informed constraints into HMMs for predictive maintenance. 
My technical background in AI/ML, coupled with my commitment to interdisciplinary problem-solving, positions me to contribute meaningfully to your research community and advance the Institute's strategic goals.

Thank you for considering my application. 
I would welcome the opportunity to discuss how my expertise in structured data transformation, 
model interpretability, and real-world AI deployment could support Fraunhofer EMI's initiatives in power network resilience. 
I look forward to contributing to groundbreaking research at the intersection of AI, infrastructure safety, and sustainable engineering.

\makeletterclosing
%%%%%%%%%%%%%%%%%%%%%%%%%%%%%%%%%%%%%%%%%%%%%%%%%%%%%%%%%%%%%%%%%%%%%%%%%%%%%%%%%%%%%%%%%%%%%%%%%%%%%%%%%%%%%%%%%%%%%
\end{document}