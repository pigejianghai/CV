\documentclass[11pt,a4paper, final]{moderncv}
\usepackage{color}
\usepackage{fontspec}
\usepackage{url}
\moderncvtheme{classic}
\definecolor{color1}{RGB}{1, 52, 64}
\usepackage[scale=0.95]{geometry}
\setlength{\hintscolumnwidth}{3.5cm} 
\AtBeginDocument{\recomputelengths} 
\usepackage{xunicode}
\usepackage{xltxtra}
\usepackage[utf8]{inputenc}
\usepackage{lipsum} 
\usepackage{tikz}
\usepackage{arydshln}
\usepackage{nccrules}
\usepackage{verbatim}
\setmainfont{Times New Roman}
\AfterPreamble{\hypersetup{
  pdfcreator={XeLaTeX},
  pdftitle={Fictional CV of Hai Jiang}
}}
\usepackage{fontawesome5}
%%%%%%%%%%%%%%%%%%%%%%%%%%%%%%%%%%%%%%%%%%%%%%%%%%%%%%%%%%%%%%%%%%%%%%%%%%%%%%%%%%%%%%%%%%%%%%%%%%%%%%%%%%%%%%%%%%%%%
%%%%%%%%%%%%%%%%%%%%%%%%%%%%%%%%%%%%%%%%%%%%%%%%%%%%%%%%%%%%%%%%%%%%%%%%%%%%%%%%%%%%%%%%%%%%%%%%%%%%%%%%%%%%%%%%%%%%%
%%%%%%%%%%%%%%%%%%%%%%%%%%%%%%%%%%%%%%%%%%%%%%%%%%%%%%%%%%%%%%%%%%%%%%%%%%%%%%%%%%%%%%%%%%%%%%%%%%%%%%%%%%%%%%%%%%%%%
%%%%%%%%%%%%%%%%%%%%%%%%%%%%%%%%%%%%%%%%%%%%%%%%%%%%%%%%%%%%%%%%%%%%%%%%%%%%%%%%%%%%%%%%%%%%%%%%%%%%%%%%%%%%%%%%%%%%%
%%%%%%%%%%%%%%%%%%%%%%%%%%%%%%%%%%%%%%%%%%%%%%%%%%%%%%%%%%%%%%%%%%%%%%%%%%%%%%%%%%%%%%%%%%%%%%%%%%%%%%%%%%%%%%%%%%%%%
\firstname{Hai}
\familyname{Jiang}
\address{Shenzhen, Guangdong, China}{}
\extrainfo{
	\faPhone\hspace{0.5em}+86 19867713757\\
	{\small\faEnvelope}\hspace{0.5em}jiangh14@lzu.edu.cn\\
}
\newcommand{\spacesection}{\vspace{0.4cm}}
\newcommand{\spacesubsection}{\vspace{0.2cm}}
%===========================
\begin{document}
\recipient{Admission Committee}
{University of Technology Nuremberg (UTN)\\
Department of Engineering\\
Natural Language Learning \& Generation Lab (NLLG)
% Science and Technology\\
% Logistics and Supply Chain Management
}
\date{\today}
\opening{Dear Members of the Admissions Committee,}
\closing{Sincerely,}
%%%%%%%%%%%%%%%%%%%%%%%%%%%%%%%%%%%%%%%%%%%%%%%%%%%%%%%%%%%%%%%%%%%%%%%%%%%%%%%%%%%%%%%%%%%%%%%%%%%%%%%%%%%%%%%%%%%%%
%%%%%%%%%%%%%%%%%%%%%%%%%%%%%%%%%%%%%%%%%%%%%%%%%%%%%%%%%%%%%%%%%%%%%%%%%%%%%%%%%%%%%%%%%%%%%%%%%%%%%%%%%%%%%%%%%%%%%
%%%%%%%%%%%%%%%%%%%%%%%%%%%%%%%%%%%%%%%%%%%%%%%%%%%%%%%%%%%%%%%%%%%%%%%%%%%%%%%%%%%%%%%%%%%%%%%%%%%%%%%%%%%%%%%%%%%%%
%%%%%%%%%%%%%%%%%%%%%%%%%%%%%%%%%%%%%%%%%%%%%%%%%%%%%%%%%%%%%%%%%%%%%%%%%%%%%%%%%%%%%%%%%%%%%%%%%%%%%%%%%%%%%%%%%%%%%
%%%%%%%%%%%%%%%%%%%%%%%%%%%%%%%%%%%%%%%%%%%%%%%%%%%%%%%%%%%%%%%%%%%%%%%%%%%%%%%%%%%%%%%%%%%%%%%%%%%%%%%%%%%%%%%%%%%%%
\makelettertitle
\thispagestyle{empty}
\pagestyle{empty}
I am writing to express my enthusiastic interest in the PhD position focused on 
\textbf{Multimodal/Multi-Agent Evaluation of Generative AI} at UTN's NLLG Lab. 
With a background in computational mathematics, hands-on experience in AI-driven research, 
and a passion for advancing robust evaluation frameworks, 
I am eager to contribute to UTN's mission of developing cutting-edge metrics for generative AI systems.

% \textbf{Academic Background \& Research Expertise}\\
During my \textbf{Master's in Computational Mathematics} (2019-2022), 
I specialized in \textbf{Generative Adversarial Networks (GANs)}, 
focusing on cross-domain style transfer. 
My thesis, \emph{``GANs for Personal Style Imitation of Chinese Handwritten Characters,''} 
involved developing an end-to-end CycleGAN-based framework to replicate the calligraphic style of Shiling Shen Chern. 
By optimizing adversarial training and domain-specific preprocessing, 
I achieved \textbf{85\% visual similarity} across 220 characters, surpassing baseline models by 10\%. 
This project honed my expertise in structured data transformation and cross-domain adaptation—skills directly applicable to knowledge graph design, where heterogeneous data integration and semantic mapping are critical.

Post-graduation, I joined Sun Yat-sen University's \textbf{Computational Medical Imaging Lab}, 
where I led AI-driven projects with clinical impact. 
For instance, I developed a \textbf{multi-task learning model} to 
classify Placenta Accreta Spectrum Disorder severity using T2-WI MRI images, achieving an \textbf{AUC of 0.80}. 
In a separate project, 
I designed a CNN-based system to predict breast cancer metastasis in Sentinel Lymph Nodes via dual-energy CT scans, 
attaining an \textbf{AUC of 0.85} in cross-validation. 
Beyond algorithmic development, I curated heterogeneous DICOM datasets and deployed models on hospital servers, 
demonstrating my ability to manage complex data pipelines—a skill vital for building robust knowledge infrastructures. 
These efforts resulted in two first-authored papers (one accepted at \textbf{ISBI 2025}, one under revision).

% \textbf{Alignment}\\
Your program's focus on \textbf{multimodal evaluation of generative AI} deeply resonates with my goal 
to advance rigorous metrics for AI systems. 
I am particularly inspired by \textbf{Prof. Dr. Steffen Eger's work} 
on enhancing the faithfulness of Natural Language Explanations (NLEs), 
which aligns with my interest in developing evaluation frameworks that ensure transparency and reliability in generative outputs. 
For instance, my CycleGAN project required quantifying stylistic fidelity—a precursor to evaluating generative models' adherence to user-defined constraints. 
Similarly, my medical imaging work involved validating models against clinician feedback, mirroring the iterative human-AI collaboration essential for refining evaluation benchmarks.

UTN's NLLG Lab, with its emphasis on \textbf{Large Language Models (LLMs)} and \textbf{multimodal systems}, 
provides an unparalleled environment to address challenges such as: 
(1) \textbf{Metric Robustness}: Enhancing evaluation tools like BERTScore and BARTScore for text-to-image generation by incorporating domain-specific semantic alignment.; 
(2) \textbf{Multi-Agent Evaluation}: Designing frameworks to assess collaborative AI systems (e.g., multi-agent chatbots) for coherence and task completion.

% \textbf{Long-Term Goals}\\
Long-term, I aspire to pioneer interpretable and scalable evaluation metrics that bridge theoretical rigor with real-world applicability, particularly in multimodal generative AI. 
This PhD will equip me with the interdisciplinary expertise—spanning machine learning, natural language processing, and human-computer interaction—to establish a research group focused on democratizing access to trustworthy AI evaluation tools. 
UTN's translational research ethos and partnerships with industry leaders will be instrumental in realizing this vision.

I am confident that my technical expertise in AI/ML, experience in managing complex data workflows, and alignment with NLLG Lab's strategic priorities position me to contribute meaningfully to your program. 
I would welcome the opportunity to discuss how my background could support initiatives such as optimizing multimodal evaluation pipelines or advancing faithfulness in generative explanations. 
Thank you for considering my application. 
I look forward to contributing to UTN's legacy of innovation in AI research.

\makeletterclosing
%%%%%%%%%%%%%%%%%%%%%%%%%%%%%%%%%%%%%%%%%%%%%%%%%%%%%%%%%%%%%%%%%%%%%%%%%%%%%%%%%%%%%%%%%%%%%%%%%%%%%%%%%%%%%%%%%%%%%
\end{document}