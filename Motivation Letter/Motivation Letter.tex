\documentclass[11pt,a4paper, final]{moderncv}
\usepackage{color}
\usepackage{fontspec}
\usepackage{url}
\moderncvtheme{classic}
\definecolor{color1}{RGB}{1, 52, 64}
\usepackage[scale=0.95]{geometry}
\setlength{\hintscolumnwidth}{3.5cm} 
\AtBeginDocument{\recomputelengths} 
\usepackage{xunicode}
\usepackage{xltxtra}
\usepackage[utf8]{inputenc}
\usepackage{lipsum} 
\usepackage{tikz}
\usepackage{arydshln}
\usepackage{nccrules}
\usepackage{verbatim}
\setmainfont{Times New Roman}
\AfterPreamble{\hypersetup{
  pdfcreator={XeLaTeX},
  pdftitle={Fictional CV of Hai Jiang}
}}
\usepackage{fontawesome5}
%%%%%%%%%%%%%%%%%%%%%%%%%%%%%%%%%%%%%%%%%%%%%%%%%%%%%%%%%%%%%%%%%%%%%%%%%%%%%%%%%%%%%%%%%%%%%%%%%%%%%%%%%%%%%%%%%%%%%
%%%%%%%%%%%%%%%%%%%%%%%%%%%%%%%%%%%%%%%%%%%%%%%%%%%%%%%%%%%%%%%%%%%%%%%%%%%%%%%%%%%%%%%%%%%%%%%%%%%%%%%%%%%%%%%%%%%%%
%%%%%%%%%%%%%%%%%%%%%%%%%%%%%%%%%%%%%%%%%%%%%%%%%%%%%%%%%%%%%%%%%%%%%%%%%%%%%%%%%%%%%%%%%%%%%%%%%%%%%%%%%%%%%%%%%%%%%
%%%%%%%%%%%%%%%%%%%%%%%%%%%%%%%%%%%%%%%%%%%%%%%%%%%%%%%%%%%%%%%%%%%%%%%%%%%%%%%%%%%%%%%%%%%%%%%%%%%%%%%%%%%%%%%%%%%%%
%%%%%%%%%%%%%%%%%%%%%%%%%%%%%%%%%%%%%%%%%%%%%%%%%%%%%%%%%%%%%%%%%%%%%%%%%%%%%%%%%%%%%%%%%%%%%%%%%%%%%%%%%%%%%%%%%%%%%
\firstname{Hai}
\familyname{Jiang}
\address{Shenzhen, China}{}
\extrainfo{
	\faPhone\hspace{0.5em}+86 19867713757\\
	{\small\faEnvelope}\hspace{0.5em}jiangh14@lzu.edu.cn\\
}
\newcommand{\spacesection}{\vspace{0.4cm}}
\newcommand{\spacesubsection}{\vspace{0.2cm}}
%===========================
\begin{document}
\recipient{Admission Committee}
{Max Planck Institute for Systems and Process Engineering\\
Sustainable Chemical Production\\
Computational Methods \& Algorithms\\
% Science and Technology\\
% Logistics and Supply Chain Management
}
\date{\today}
\opening{Dear Members of the Admissions Committee,}
\closing{Sincerely,}
%%%%%%%%%%%%%%%%%%%%%%%%%%%%%%%%%%%%%%%%%%%%%%%%%%%%%%%%%%%%%%%%%%%%%%%%%%%%%%%%%%%%%%%%%%%%%%%%%%%%%%%%%%%%%%%%%%%%%
%%%%%%%%%%%%%%%%%%%%%%%%%%%%%%%%%%%%%%%%%%%%%%%%%%%%%%%%%%%%%%%%%%%%%%%%%%%%%%%%%%%%%%%%%%%%%%%%%%%%%%%%%%%%%%%%%%%%%
%%%%%%%%%%%%%%%%%%%%%%%%%%%%%%%%%%%%%%%%%%%%%%%%%%%%%%%%%%%%%%%%%%%%%%%%%%%%%%%%%%%%%%%%%%%%%%%%%%%%%%%%%%%%%%%%%%%%%
%%%%%%%%%%%%%%%%%%%%%%%%%%%%%%%%%%%%%%%%%%%%%%%%%%%%%%%%%%%%%%%%%%%%%%%%%%%%%%%%%%%%%%%%%%%%%%%%%%%%%%%%%%%%%%%%%%%%%
%%%%%%%%%%%%%%%%%%%%%%%%%%%%%%%%%%%%%%%%%%%%%%%%%%%%%%%%%%%%%%%%%%%%%%%%%%%%%%%%%%%%%%%%%%%%%%%%%%%%%%%%%%%%%%%%%%%%%
\makelettertitle
\thispagestyle{empty}
\pagestyle{empty}
I am writing to express my enthusiastic interest in the PhD position 
in \textbf{Numerical Mathematics with Applications} at the Max Planck Institute. 
With a robust foundation in computational mathematics, hands-on experience in AI-driven medical image analysis, 
and a passion for applying machine learning to complex systems, 
I am eager to contribute to your mission of advancing virtual process models and computational methodologies for sustainable chemical production.

% \textbf{Academic Background \& Research Expertise}\\
During my Master's in Computational Mathematics (2019-2022), 
I specialized in \textbf{Generative Adversarial Networks (GANs)}, 
focusing on cross-domain style transfer for structured data. 
My thesis, \emph{``GANs for Personal Style Imitation of Chinese Handwritten Characters,''} 
nvolved designing an end-to-end CycleGAN framework to replicate the nuanced calligraphic style of Shiling Shen Chern. 
By optimizing adversarial training and domain-specific preprocessing, 
I achieved \textbf{85\% visual similarity} across 220 characters, outperforming baseline models by 10\%. 
This project honed my ability to map heterogeneous data domains—a skill directly transferable to integrating multi-scale chemical process data into knowledge graphs or hybrid models.

Post-graduation, I joined Sun Yat-sen University's \textbf{Computational Medical Imaging Lab}, 
where I led AI-driven projects with clinical impact. 
For instance, I developed a \textbf{multi-task learning model} to 
classify Placenta Accreta Spectrum Disorder severity using T2-WI MRI images, achieving an \textbf{AUC of 0.80}. 
In a separate project, 
I designed a CNN-based system to predict breast cancer metastasis in Sentinel Lymph Nodes via dual-energy CT scans, 
attaining an \textbf{AUC of 0.85} in cross-validation. 
Beyond algorithmic development, I curated heterogeneous DICOM datasets and deployed models on hospital servers, 
demonstrating my ability to manage complex data pipelines—a skill vital for building robust knowledge infrastructures. 
These efforts resulted in two first-authored papers (one accepted at \textbf{ISBI 2025}, one under revision).

% \textbf{Alignment}\\
Your program's emphasis on \textbf{digitalization and machine learning in the future of complex chemical production} 
deeply resonates with my aspiration to develop interpretable, physics-informed algorithms for tackling complexity. 
I am particularly inspired by \textbf{Prof. Dr. Thomas Richter's work} 
on hybrid models integrating partial differential equations (PDEs) with neural networks. 
My experience with GANs for structured data transformation and CNNs for medical imaging aligns with the challenge of accelerating numerical simulations of coupled PDE systems. 
For instance, I envision leveraging my expertise in adversarial training to enhance neural network-based surrogate models, 
enabling faster optimization of multiphase reaction processes while preserving interpretability.

A PhD at MPI would provide the ideal interdisciplinary environment—spanning machine learning, computational mathematics, 
and chemistry—to advance my goal of pioneering \textbf{Numerical Mathematics with Applications} 
that bridge theoretical rigor and industrial relevance. 
Long-term, I aim to establish a research group focused on 
AI-driven numerical optimization for multiphysics systems, 
addressing challenges such as catalyst design or reactor scalability. 
The Institute's translational ethos and partnerships with industry leaders would empower me to drive innovations from theory to real-world impact, aligning with MPI's legacy in efficient simulation methodologies.

% \textbf{Long-Term Goals}\\
The IMPRS program's focus on scientific machine learning and temporal multiscale modeling offers unparalleled opportunities to refine my skills in computational method development. 
I am eager to collaborate on projects such as adaptive surrogate modeling for dynamic chemical processes or knowledge graph-driven optimization frameworks. 
My technical background, coupled with my commitment to interdisciplinary problem-solving, positions me to contribute meaningfully to your research community.

Thank you for considering my application. I would welcome the opportunity to discuss how my expertise in AI/ML, computational mathematics, 
and data-driven modeling could support MPI's strategic initiatives. 
I look forward to contributing to groundbreaking research at the intersection of algorithms, chemistry, and sustainability.

\makeletterclosing
%%%%%%%%%%%%%%%%%%%%%%%%%%%%%%%%%%%%%%%%%%%%%%%%%%%%%%%%%%%%%%%%%%%%%%%%%%%%%%%%%%%%%%%%%%%%%%%%%%%%%%%%%%%%%%%%%%%%%
\end{document}