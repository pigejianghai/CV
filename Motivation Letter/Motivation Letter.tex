\documentclass[11pt,a4paper, final]{moderncv}
\usepackage{color}
\usepackage{fontspec}
\usepackage{url}
\moderncvtheme{classic}
\definecolor{color1}{RGB}{1, 52, 64}
\usepackage[scale=0.8]{geometry}
\setlength{\hintscolumnwidth}{3.5cm} 
\AtBeginDocument{\recomputelengths} 
\usepackage{xunicode}
\usepackage{xltxtra}
\usepackage[utf8]{inputenc}
\usepackage{lipsum} 
\usepackage{tikz}
\usepackage{arydshln}
\usepackage{nccrules}
\usepackage{verbatim}
\setmainfont{Times New Roman}
\AfterPreamble{\hypersetup{
  pdfcreator={XeLaTeX},
  pdftitle={Fictional CV of Hai Jiang}
}}
\usepackage{fontawesome5}
%%%%%%%%%%%%%%%%%%%%%%%%%%%%%%%%%%%%%%%%%%%%%%%%%%%%%%%%%%%%%%%%%%%%%%%%%%%%%%%%%%%%%%%%%%%%%%%%%%%%%%%%%%%%%%%%%%%%%
%%%%%%%%%%%%%%%%%%%%%%%%%%%%%%%%%%%%%%%%%%%%%%%%%%%%%%%%%%%%%%%%%%%%%%%%%%%%%%%%%%%%%%%%%%%%%%%%%%%%%%%%%%%%%%%%%%%%%
%%%%%%%%%%%%%%%%%%%%%%%%%%%%%%%%%%%%%%%%%%%%%%%%%%%%%%%%%%%%%%%%%%%%%%%%%%%%%%%%%%%%%%%%%%%%%%%%%%%%%%%%%%%%%%%%%%%%%
%%%%%%%%%%%%%%%%%%%%%%%%%%%%%%%%%%%%%%%%%%%%%%%%%%%%%%%%%%%%%%%%%%%%%%%%%%%%%%%%%%%%%%%%%%%%%%%%%%%%%%%%%%%%%%%%%%%%%
%%%%%%%%%%%%%%%%%%%%%%%%%%%%%%%%%%%%%%%%%%%%%%%%%%%%%%%%%%%%%%%%%%%%%%%%%%%%%%%%%%%%%%%%%%%%%%%%%%%%%%%%%%%%%%%%%%%%%
\firstname{Hai}
\familyname{Jiang}
\address{Shenzhen, China}{}
\extrainfo{
	\faPhone\hspace{0.5em}+86 19867713757\\
	{\small\faEnvelope}\hspace{0.5em}jiangh14@lzu.edu.cn\\
}
\newcommand{\spacesection}{\vspace{0.4cm}}
\newcommand{\spacesubsection}{\vspace{0.2cm}}
%===========================
\begin{document}
\recipient{Admission Committee}
{University of Freiburg\\
Department of Sustainable Systems Engineering\\
Marie Skłodowska-Curie Actions (MSCA)
% System State Monitoring (STM) \\
% Science and Technology\\
% Logistics and Supply Chain Management
}
\date{\today}
\opening{Dear Members of the Admissions Committee,}
\closing{Sincerely,}
%%%%%%%%%%%%%%%%%%%%%%%%%%%%%%%%%%%%%%%%%%%%%%%%%%%%%%%%%%%%%%%%%%%%%%%%%%%%%%%%%%%%%%%%%%%%%%%%%%%%%%%%%%%%%%%%%%%%%
%%%%%%%%%%%%%%%%%%%%%%%%%%%%%%%%%%%%%%%%%%%%%%%%%%%%%%%%%%%%%%%%%%%%%%%%%%%%%%%%%%%%%%%%%%%%%%%%%%%%%%%%%%%%%%%%%%%%%
%%%%%%%%%%%%%%%%%%%%%%%%%%%%%%%%%%%%%%%%%%%%%%%%%%%%%%%%%%%%%%%%%%%%%%%%%%%%%%%%%%%%%%%%%%%%%%%%%%%%%%%%%%%%%%%%%%%%%
%%%%%%%%%%%%%%%%%%%%%%%%%%%%%%%%%%%%%%%%%%%%%%%%%%%%%%%%%%%%%%%%%%%%%%%%%%%%%%%%%%%%%%%%%%%%%%%%%%%%%%%%%%%%%%%%%%%%%
%%%%%%%%%%%%%%%%%%%%%%%%%%%%%%%%%%%%%%%%%%%%%%%%%%%%%%%%%%%%%%%%%%%%%%%%%%%%%%%%%%%%%%%%%%%%%%%%%%%%%%%%%%%%%%%%%%%%%
\makelettertitle
\thispagestyle{empty}
\pagestyle{empty}
I am writing to express my enthusiastic interest in the PhD position 
in \textbf{``Super-resolution techniques to enhance low-resolution metering and inspection data''} (FOURIER DC 5) 
at the University of Freiburg. 
With a robust foundation in computational mathematics, 
hands-on experience in AI-driven structured data transformation, 
and a passion for developing scalable methodologies to optimize infrastructure monitoring, 
I am eager to contribute to MSCA's vision of advancing greener, resilient, 
and intelligent societies through innovative engineering solutions.

% \textbf{Academic Background \& Research Expertise}\\
During my Master's in Computational Mathematics (2019-2022), 
I specialized in \textbf{Generative Adversarial Networks (GANs)}, 
focusing on cross-domain style transfer for structured data. 
My thesis, \emph{``GANs for Personal Style Imitation of Chinese Handwritten Characters,''} 
nvolved designing an end-to-end CycleGAN framework to replicate the nuanced calligraphic style of Shiling Shen Chern. 
By optimizing adversarial training and domain-specific preprocessing, 
I achieved \textbf{85\% visual similarity} across 220 characters, outperforming baseline models by 10\%. 
This project honed my ability to map heterogeneous data domains—a skill directly applicable to enhancing low-resolution infrastructure data through super-resolution techniques.
For instance, the adversarial training principles I employed could be adapted to reconstruct high-fidelity metering data from sparse inputs, enabling precise anomaly detection in complex systems.

As a researcher at Sun Yat-sen University's \textbf{Computational Medical Imaging Lab}, 
I led interdisciplinary projects that combined AI innovation with clinical relevance. 
For example, I developed a \textbf{multi-task learning model} to 
classify Placenta Accreta Spectrum Disorder severity using T2-WI MRI images, achieving an \textbf{AUC of 0.80}. 
In a separate project, 
I designed a CNN-based system to predict breast cancer metastasis in Sentinel Lymph Nodes via dual-energy CT scans, 
attaining an \textbf{AUC of 0.85} in cross-validation. 
Beyond algorithm development, and harmonized heterogeneous DICOM datasets while deploying models on hospital servers, 
demonstrating my proficiency in managing complex data workflows—a critical competency for ensuring robust super-resolution pipelines in infrastructure monitoring. 
These efforts resulted in two first-authored papers (one accepted at ISBI 2025, one under revision).

% \textbf{Alignment}\\
Your program's emphasis on \textbf{developing super-resolution techniques to enhance low-resolution metering and inspection data} 
aligns with my aspiration to develop efficient, interpretable AI frameworks for large-scale infrastructure resilience. 
My expertise in GANs for structured data transformation and CNNs for medical imaging directly supports the challenge of reconstructing high-resolution insights from sparse or noisy sensor data. 
For instance, the domain adaptation strategies I employed in medical imaging could be repurposed to align disparate sensor data distributions in energy grids or transportation networks, enhancing the reliability of predictive maintenance systems. 
Furthermore, my experience in deploying clinically validated models underscores my ability to bridge theoretical AI advancements with practical engineering applications—key to achieving MSCA's goals of smart, sustainable societies.

A PhD at the University of Freiburg would provide the ideal interdisciplinary environment—spanning AI, engineering, and applied mathematics—to advance my goal of pioneering interpretable super-resolution frameworks for infrastructure monitoring. 
Long-term, I aim to lead research focused on adaptive algorithms that integrate multi-modal sensor data (e.g., thermal, acoustic, and visual inputs) to enable real-time anomaly detection and proactive maintenance. 
The University's translational ethos, state-of-the-art facilities, and collaborations with institutions like the Fraunhofer Institute for High-Speed Dynamics would empower me to validate these innovations in real-world scenarios, aligning with Freiburg's legacy in sustainable engineering.

% \textbf{Long-Term Goals}\\
The MSCA program's focus on AI-driven infrastructure resilience offers unparalleled opportunities to refine my expertise in computational method development. 
I am particularly excited to collaborate on projects such as physics-informed neural networks for dynamic system modeling or federated learning frameworks for distributed sensor networks. 
My technical background in AI/ML, coupled with my commitment to interdisciplinary problem-solving, positions me to contribute meaningfully to the FOURIER DC 5 project and advance the University's strategic objectives in sustainable systems engineering.

Thank you for considering my application. 
I would welcome the opportunity to discuss how my expertise in adversarial training, data harmonization, and cross-domain adaptation could support MSCA's mission. 
I look forward to contributing to groundbreaking research at the intersection of AI, super-resolution techniques, and resilient infrastructure design.

\makeletterclosing
%%%%%%%%%%%%%%%%%%%%%%%%%%%%%%%%%%%%%%%%%%%%%%%%%%%%%%%%%%%%%%%%%%%%%%%%%%%%%%%%%%%%%%%%%%%%%%%%%%%%%%%%%%%%%%%%%%%%%
\end{document}