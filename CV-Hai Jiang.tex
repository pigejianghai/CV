\documentclass[11pt,a4paper, final]{moderncv}
\usepackage{color}
\usepackage{fontspec}
\moderncvtheme{classic}
\definecolor{color1}{RGB}{1, 52, 64}
\usepackage[scale=0.935]{geometry}
\setlength{\hintscolumnwidth}{3.5cm} 
\AtBeginDocument{\recomputelengths} 
\usepackage{xunicode}
\usepackage{xltxtra}
\usepackage[utf8]{inputenc}
\usepackage{lipsum} 
\usepackage{tikz}
\usepackage{verbatim}
\usepackage[
	sorting=none,
	minbibnames=8,
	maxbibnames=9,
	block=space
]{biblatex}
\bibliography{publications}
\defbibenvironment{midbib}
{\list
	{}
	{
		\setlength{\leftmargin}{0mm}
		\setlength{\itemindent}{-\leftmargin}
		\setlength{\itemsep}{\bibitemsep}
		\setlength{\parsep}{\bibparsep}}
	}
	{\endlist}
{\item}
\DeclareSourcemap{
	\maps[datatype=bibtex]{
		\map{
			\step[fieldsource=mydoi]
			\step[fieldset=usera, origfieldval]
		}
		\map{
			\step[fieldsource=mypdf]
			\step[fieldset=userb, origfieldval]
		}
	}
}
\DeclareFieldFormat{usera}{\color{color1}[\href{#1}{\textsc{doi}}]}
\AtEveryBibitem{
	\csappto{blx@bbx@\thefield{entrytype}}{% 
		\iffieldundef{usera}{ 
			% this gets invoked, once nothing is supplied
			% via the mypdf or mydoi value.
			% you could e.g. display a default thing here.
		}{\space\printfield{usera}}}
}
\DeclareFieldFormat{userb}{\color{color1}[\href{#1}{\textsc{pdf}}]}
\AtEveryBibitem{
	\csappto{blx@bbx@\thefield{entrytype}}{\iffieldundef{userb}{}{\printfield{userb}}}
}
\renewcommand*{\mkbibnamegiven}[1]{%
	\ifitemannotation{highlight}
	{\textbf{#1}}
	{#1}}

	\renewcommand*{\mkbibnamefamily}[1]{%
	\ifitemannotation{highlight}
	{\textbf{#1}}
	{#1}
}
\setmainfont{Times New Roman}%[Numbers=OldStyle]
\AfterPreamble{\hypersetup{
  pdfcreator={XeLaTeX},
  pdftitle={Fictional CV of Hai Jiang}
}}
\usepackage{fontawesome5}
\firstname{Hai}
\familyname{Jiang}
\address{Shenzhen, Guangdong, China}{}
\extrainfo{
	\faPhone\hspace{0.5em}+86 19867713757\\
	{\small\faEnvelope}\hspace{0.5em}jiangh14@lzu.edu.cn
}
\newcommand{\spacesection}{\vspace{0.4cm}}
\newcommand{\spacesubsection}{\vspace{0.2cm}}
%===========================
\begin{document}
\recipient{Admission Committee}
{University of Twente\\
Faculty of Electrical Engineering, Mathematics and Computer Science \\ 
Computer Science \\ 
Pervasive Systems}
\date{\today}
\opening{To whom it may concern,}
\closing{Sincerely,}
% \enclosure[Attached]{curriculum vit\ae{}}
%%%%%%%%%%%%%%%%%%%%%%%%%%%%%%%%%%%%%%%%%%%%%%%%%%%%%%%%%%%%%%%%%%%%%%%%%%%%%%%%%%%%%%%%%%%%%%%%%%%%%%%%%%%%%%%%%%%%%
%%%%%%%%%%%%%%%%%%%%%%%%%%%%%%%%%%%%%%%%%%%%%%%%%%%%%%%%%%%%%%%%%%%%%%%%%%%%%%%%%%%%%%%%%%%%%%%%%%%%%%%%%%%%%%%%%%%%%
%%%%%%%%%%%%%%%%%%%%%%%%%%%%%%%%%%%%%%%%%%%%%%%%%%%%%%%%%%%%%%%%%%%%%%%%%%%%%%%%%%%%%%%%%%%%%%%%%%%%%%%%%%%%%%%%%%%%%
%%%%%%%%%%%%%%%%%%%%%%%%%%%%%%%%%%%%%%%%%%%%%%%%%%%%%%%%%%%%%%%%%%%%%%%%%%%%%%%%%%%%%%%%%%%%%%%%%%%%%%%%%%%%%%%%%%%%%
%%%%%%%%%%%%%%%%%%%%%%%%%%%%%%%%%%%%%%%%%%%%%%%%%%%%%%%%%%%%%%%%%%%%%%%%%%%%%%%%%%%%%%%%%%%%%%%%%%%%%%%%%%%%%%%%%%%%%
\makelettertitle
I am writing to express my interest in the PhD position at your university, 
specializing in \textbf{Multi-modal Information Fusion in Dynamic Environments}. 
My academic background and research experience have prepared me well for advanced study in this field, 
and I am highly motivated to contributed to cutting-edge research that combines AI with Mathematics. 

% \spacesubsection
My journey began with a strong foundation in Computer Science and Mathematics, 
leading me to pursue a Master's degree in Computational Mathematics in 2019, 
following a Bachelor's degree in Information Security in China. 
For my Master's project, I focused on Generative Adversarial Networks (GANs) of Deep Learning (DL), 
specifically for handwriting style imitation. 
In this project, I developed a GANs-based model to replicate individual handwriting styles, 
focusing on emulating the handwritten Chinese characters of Shiing Shen Chern from a dataset of around 220 characters. 
This experience provided me with hands-on expertise in GANs, 
data preprocessing, manuscript analysis, and code replication, 
culminating in my Master's thesis, \emph{“GANs based Personal Style Imitation of Chinese Handwritten Characters.”}

% \spacesubsection
Following my Master's degree, 
I contributed to the Compressed Sensing MRI ADMM-Net project, 
which combined DL with numerical approximation theory. 
Traditional algorithms often lose image detail with repeated iterations, 
but by leveraging Convolutional Neural Networks (CNNs), 
our team demonstrated how DL can dynamically adjust parameters to preserve fine details. 
This project deepened my understanding of convergent algorithm theory, proof construction, 
and the balance between theoretical approaches and practical applications, 
while refining my programming and analytical skills.

% \spacesubsection
After completing my degree, 
I joined a research group at Sun Yat-sen University focused on Computer-aided diagnosis, 
where I worked as a Research Assistant on several impactful projects. 
These included diagnosing Placenta Accreta Spectrum Disorders, 
predicting metastasis in Sentinel Axillary Lymph Nodes in breast cancer, 
and assessing responses to Neoadjuvant Chemotherapy via MRI. 
Through these projects, I gained further expertise in Python, PyTorch, and TensorFlow, 
along with experience in manuscript research and scientific writing.

% \spacesubsection
I am confident that I am an excellent candidate for this position within your research group. 
Your focus on computer vision, machine learning, and their intersection for human behaviour understanding, 
closely aligns with my research interests. 
I am especially passionate about advancing multi-modal data analysis and developing innovative algorithms 
for emotion recognition by analyzing verbal and non-verbal social communication cues in videos. 
My interdisciplinary background, programming expertise, and research experience uniquely equip me 
to contribute to and excel in the challenges and rigor of this PhD position.

% \spacesubsection
My long-term goal is to establish a career in academia, 
where I can contribute to the fields of Computer Science and Mathematics through meaningful research and innovation. 
The interdisciplinary nature of your group, alongside the expertise and contributions of its members, 
closely matches my academic and professional aspirations, making this position an ideal next step. 
I am eager to join an environment that fosters intellectual rigor and collaboration, 
providing the resources essential to realizing my goals.

% \spacesubsection
I am enthusiastic about the opportunity to join your research community 
and contribute actively while continuing to develop my expertise. 
Thank you for considering my application, 
and I look forward to discussing how my background, skills, and goals align with your program.

% \spacesubsection

\makeletterclosing

\clearpage
%%%%%%%%%%%%%%%%%%%%%%%%%%%%%%%%%%%%%%%%%%%%%%%%%%%%%%%%%%%%%%%%%%%%%%%%%%%%%%%%%%%%%%%%%%%%%%%%%%%%%%%%%%%%%%%%%%%%%
%%%%%%%%%%%%%%%%%%%%%%%%%%%%%%%%%%%%%%%%%%%%%%%%%%%%%%%%%%%%%%%%%%%%%%%%%%%%%%%%%%%%%%%%%%%%%%%%%%%%%%%%%%%%%%%%%%%%%
%%%%%%%%%%%%%%%%%%%%%%%%%%%%%%%%%%%%%%%%%%%%%%%%%%%%%%%%%%%%%%%%%%%%%%%%%%%%%%%%%%%%%%%%%%%%%%%%%%%%%%%%%%%%%%%%%%%%%
%%%%%%%%%%%%%%%%%%%%%%%%%%%%%%%%%%%%%%%%%%%%%%%%%%%%%%%%%%%%%%%%%%%%%%%%%%%%%%%%%%%%%%%%%%%%%%%%%%%%%%%%%%%%%%%%%%%%%
%%%%%%%%%%%%%%%%%%%%%%%%%%%%%%%%%%%%%%%%%%%%%%%%%%%%%%%%%%%%%%%%%%%%%%%%%%%%%%%%%%%%%%%%%%%%%%%%%%%%%%%%%%%%%%%%%%%%%
\maketitle
%%%%%%%%%%%%%%%%%%%%%%%%%%%%%%%%%%%%%%%%%%%%%%%%%%%%%%%%%%%%%%%%%%%%%%%%%%%%%%%%%%%%%%%%%%%%%%%%%%%%%%%%%%%%%%%%%%%%%
%%%%%%%%%%%%%%%%%%%%%%%%%%%%%%%%%%%%%%%%%%%%%%%%%%%%%%%%%%%%%%%%%%%%%%%%%%%%%%%%%%%%%%%%%%%%%%%%%%%%%%%%%%%%%%%%%%%%%
%%%%%%%%%%%%%%%%%%%%%%%%%%%%%%%%%%%%%%%%%%%%%%%%%%%%%%%%%%%%%%%%%%%%%%%%%%%%%%%%%%%%%%%%%%%%%%%%%%%%%%%%%%%%%%%%%%%%%
%%%%%%%%%%%%%%%%%%%%%%%%%%%%%%%%%%%%%%%%%%%%%%%%%%%%%%%%%%%%%%%%%%%%%%%%%%%%%%%%%%%%%%%%%%%%%%%%%%%%%%%%%%%%%%%%%%%%%
%%%%%%%%%%%%%%%%%%%%%%%%%%%%%%%%%%%%%%%%%%%%%%%%%%%%%%%%%%%%%%%%%%%%%%%%%%%%%%%%%%%%%%%%%%%%%%%%%%%%%%%%%%%%%%%%%%%%%
\section{\textbf{Education}}
	\cventry{\textbf{2019--2022}}{Master of Science in Computational Mathematics}
	{\textbf{Nankai University (NKU)}}{}{\textbf{China}}{}
	\cvline{\textbf{Thesis}}
	{\emph{GANs based Personal Style Imitation of Chinese Handwritten Characters.}}
	\cvline{\textbf{Advisors}}
	{Prof.~Yunhua~Xue, Prof.~Chunlin~Wu}
	\cvline{\textbf{Related Courses}}
	{Approximation Theory and Methods, Numerical Optimization, Convex Analysis, 
	Variational Analysis, Real Analysis, Functional Analysis, Matrix Computation, 
	Foundations of Measure Theory and Probability, Numerical Solutions of Partial Differential Equations, and more.}
	\cvline{\textbf{Cumulative GPA}}
	{3.06/4.00}
	\cventry{\textbf{2014--2018}}{Bachelor of Engineering in Information Security}
	{\textbf{Lanzhou University (LZU)}}{}{\textbf{China}}{}
	\cvline{\textbf{Thesis}}
	{\emph{Improved Upper Bounds of Roman Domination Number in Maximal Outerplanar Graphs.}}
	\cvline{\textbf{Advisor}}
	{Prof.~Zepeng~Li}
	\cvline{\textbf{Related Courses}}
	{Discrete Mathematics, Operating Systems, Data Structures, C and C++ Programming Lab, 
	Java Programming Lab, Database Theory and Lab, Computer Organization and Design, and more.}
	\cvline{\textbf{Cumulative GPA}}
	{4.15/5.00}
%%%%%%%%%%%%%%%%%%%%%%%%%%%%%%%%%%%%%%%%%%%%%%%%%%%%%%%%%%%%%%%%%%%%%%%%%%%%%%%%%%%%%%%%%%%%%%%%%%%%%%%%%%%%%%%%%%%%%
%%%%%%%%%%%%%%%%%%%%%%%%%%%%%%%%%%%%%%%%%%%%%%%%%%%%%%%%%%%%%%%%%%%%%%%%%%%%%%%%%%%%%%%%%%%%%%%%%%%%%%%%%%%%%%%%%%%%%
%%%%%%%%%%%%%%%%%%%%%%%%%%%%%%%%%%%%%%%%%%%%%%%%%%%%%%%%%%%%%%%%%%%%%%%%%%%%%%%%%%%%%%%%%%%%%%%%%%%%%%%%%%%%%%%%%%%%%
%%%%%%%%%%%%%%%%%%%%%%%%%%%%%%%%%%%%%%%%%%%%%%%%%%%%%%%%%%%%%%%%%%%%%%%%%%%%%%%%%%%%%%%%%%%%%%%%%%%%%%%%%%%%%%%%%%%%%
%%%%%%%%%%%%%%%%%%%%%%%%%%%%%%%%%%%%%%%%%%%%%%%%%%%%%%%%%%%%%%%%%%%%%%%%%%%%%%%%%%%%%%%%%%%%%%%%%%%%%%%%%%%%%%%%%%%%%
\section{\textbf{Research Experience}}%, Sun Yat-sen University}
	\cventry{\textbf{11.2022--07.2024}}{Research Assistant}{Computational Medical Imaging Laboratory}
	{}{}{School of Computer Science and Engineering, Sun Yat-sen University, Guangzhou, China}
	\cvline{Project}
	{China Department of Science and Technology Key Grant, focused on Breast Cancer, 
	aims to develop models with clinical interpretability and generalization.}
	\cvline{Correspondence}{Prof.~Yao~Lu, Dr.~Ting~Song}
	\cvline{Task Focus}{Placenta Accreta Spectrum Disorders, T2-WI MRI, Prenatal Diagnosis, Multi-class classification.}
	\cvline{Experience and Skills}{Literature research, data preprocessing, model building (programming), research paper writing.}
	\cvline{Publication}
	{Submitted to ISBI 2025 and currently under review: 
	\emph{“Anatomy-guided Multitask Learning for MRI-based Classification of Placenta Accreta Spectrum and its Subtypes.”}}
	\cventry{\textbf{12.2023--01.2024}}{Research Assistant}{Computational Medical Imaging Laboratory}
	{}{}{School of Computer Science and Engineering, Sun Yat-sen University, Guangzhou, China}
	\cvline{Project}
	{National Natural Science Foundation of China, focused on Breast Cancer, 
	aimed to develop a prediction model for the Chinese female population mainly with FFDM and US.}
	\cvline{Correspondence}{Prof.~Yao~Lu, Dr.~Xiang~Zhang}
	\cvline{Task Focus}{Breast Cancer, Dual-Energy CT, Sentinel Lymph Nodes, Metestatic status, Multi-class classification.}
	\cvline{Experience and Skills}
	{The first comprehensive research experience involved conducting literature reviews, designing experiments, 
	writing research papers, and working with the TensorFlow and Keras frameworks.}
	\cvline{Publication}
	{Submitted to MICCAI 2024 and revised for submission to the Journal of \emph{Medical Physics}: 
	\emph{“DECT-Based Space-Squeeze Method for Multi-Class Classification of Metastatic Lymph Nodes in Breast Cancer.”}}
% \subsection{Multi-view based Atypical Architectural Distortion Detection of Breast Cancer}
% \cventry{\textbf{08.2022--03.2023}}
% {Advisor: Yao Lu, Cooperator: Gary Pan}
% {Sun Yat-sen University}
% {Submitted to the Journal of Physics in Medicine and Biology}{Accepted}
% {
% ■	A task of the project National Natural Science Foundation of China(NSFC) under Grant 12126610, is in progress. 
% The project aimed to develop a personalized prediction model for early breast cancer prevention in the Chinese female population. \\
% ■	The task aims to detect the breast architectural distortion based on the DBT(Digital Breast Tomosynthesis) volume with two different angles. 
% Inspired by the Siamese-Networks initially used in face recognition, the task utilized the Siamese architecture to extract and compare information from both views. 
% Experimental results show that the corresponding triplet module did help the False-Positive reduction from two angles.\\
% ■	My involvement in the project focused on preparatory works including providing ideas and consultation.\\
% ■	I gained my knowledge in multiple machine learning methods, 
% including automatic assessment of architectural distortion, detection technology in deep learning, et al.\\
% }
%%%%%%%%%%%%%%%%%%%%%%%%%%%%%%%%%%%%%%%%%%%%%%%%%%%%%%%%%%%%%%%%%%%%%%%%%%%%%%%%%%%%%%%%%%%%%%%%%%%%%%%%%%%%%%%%%%%%%
%%%%%%%%%%%%%%%%%%%%%%%%%%%%%%%%%%%%%%%%%%%%%%%%%%%%%%%%%%%%%%%%%%%%%%%%%%%%%%%%%%%%%%%%%%%%%%%%%%%%%%%%%%%%%%%%%%%%%
%%%%%%%%%%%%%%%%%%%%%%%%%%%%%%%%%%%%%%%%%%%%%%%%%%%%%%%%%%%%%%%%%%%%%%%%%%%%%%%%%%%%%%%%%%%%%%%%%%%%%%%%%%%%%%%%%%%%%
%%%%%%%%%%%%%%%%%%%%%%%%%%%%%%%%%%%%%%%%%%%%%%%%%%%%%%%%%%%%%%%%%%%%%%%%%%%%%%%%%%%%%%%%%%%%%%%%%%%%%%%%%%%%%%%%%%%%%
%%%%%%%%%%%%%%%%%%%%%%%%%%%%%%%%%%%%%%%%%%%%%%%%%%%%%%%%%%%%%%%%%%%%%%%%%%%%%%%%%%%%%%%%%%%%%%%%%%%%%%%%%%%%%%%%%%%%%
% \section{\textbf{Research Experience I}, Nankai University}
	\cventry{\textbf{01.2022--06.2022}}{Research Student}{Image Analysis Team}
	{}{}{School of Mathematical Sciences, Nankai University, Tianjin, China}
	\cvline{Task}
	{ADMM model from the manuscript “Deep ADMM-Net for Compressed-Sensing MRI.”}
	\cvline{Supervisors}{Prof.~Chunlin~Wu, Prof.~Yunhua~Xue}
	\cvline{Focus}{Compressed-sensing Theory, Iterative Equations, Neural Networks, MRI reconstruction.}
	\cvline{Experience and Skills}
	{The second programming experience involved proving mathematical equations and applying Deep Learning techniques. 
	I reproduced the iterative mathematical equations using C++, Python, and PyTorch.}
	\cventry{\textbf{01.2021--04.2021}}{Research Student}{Image Analysis Team}
	{}{}{School of Mathematical Sciences, Nankai University, Tianjin, China}
	\cvline{Task}
	{ROF-model from the manuscript “Nonlinear Total Variation Based Noise Removal Algorithms.”}
	\cvline{Supervisor}{Prof.~Yunhua~Xue}
	\cvline{Focus}{Image Restoration, Denoise, PDE, Total-Variation Penalty.}
	\cvline{Experience and Skills}
	{My initial project experience included proving mathematical equations and using both C++ and Python to develop the ROF model.}
%%%%%%%%%%%%%%%%%%%%%%%%%%%%%%%%%%%%%%%%%%%%%%%%%%%%%%%%%%%%%%%%%%%%%%%%%%%%%%%%%%%%%%%%%%%%%%%%%%%%%%%%%%%%%%%%%%%%%
%%%%%%%%%%%%%%%%%%%%%%%%%%%%%%%%%%%%%%%%%%%%%%%%%%%%%%%%%%%%%%%%%%%%%%%%%%%%%%%%%%%%%%%%%%%%%%%%%%%%%%%%%%%%%%%%%%%%%
%%%%%%%%%%%%%%%%%%%%%%%%%%%%%%%%%%%%%%%%%%%%%%%%%%%%%%%%%%%%%%%%%%%%%%%%%%%%%%%%%%%%%%%%%%%%%%%%%%%%%%%%%%%%%%%%%%%%%
%%%%%%%%%%%%%%%%%%%%%%%%%%%%%%%%%%%%%%%%%%%%%%%%%%%%%%%%%%%%%%%%%%%%%%%%%%%%%%%%%%%%%%%%%%%%%%%%%%%%%%%%%%%%%%%%%%%%%
%%%%%%%%%%%%%%%%%%%%%%%%%%%%%%%%%%%%%%%%%%%%%%%%%%%%%%%%%%%%%%%%%%%%%%%%%%%%%%%%%%%%%%%%%%%%%%%%%%%%%%%%%%%%%%%%%%%%%
% \newpage
% \spacesection
\section{\textbf{Other Work Experience}}
	\subsection{\textbf{Funding}}
		\cvline{\textbf{Proposal Writing}}
		{Accepted; China Department of Science and Technology Key Grant 2023YFE0204300.}
		\cvline{\textbf{Report Writing}}
		{Succeeded; Finished three Completion Reports and three Progress Reports; the NSFC Grant 81971691, 12126610, 
		the R\&D project of Pazhou Lab (Huangpu) under Grant 2023K0606.}
		% \spacesubsection
	\subsection{\textbf{Specification}}
		\cvline{\textbf{Patent}}
		{1 Patent Application Specification; under review.}
		\cvline{\textbf{Device}}
		{1 Medical Device Application Specification; succeeded.}
		% \spacesubsection
	\subsection{\textbf{Teaching Experience}}
		\cvline{\textbf{Courses}}{Calculus; Mathematical Analysis}
		\cvline{\textbf{Thesis}}{\emph{Breast Cancer Classification Method Based on Dual-Energy CT Images.}}
%%%%%%%%%%%%%%%%%%%%%%%%%%%%%%%%%%%%%%%%%%%%%%%%%%%%%%%%%%%%%%%%%%%%%%%%%%%%%%%%%%%%%%%%%%%%%%%%%%%%%%%%%%%%%%%%%%%%%
%%%%%%%%%%%%%%%%%%%%%%%%%%%%%%%%%%%%%%%%%%%%%%%%%%%%%%%%%%%%%%%%%%%%%%%%%%%%%%%%%%%%%%%%%%%%%%%%%%%%%%%%%%%%%%%%%%%%%
%%%%%%%%%%%%%%%%%%%%%%%%%%%%%%%%%%%%%%%%%%%%%%%%%%%%%%%%%%%%%%%%%%%%%%%%%%%%%%%%%%%%%%%%%%%%%%%%%%%%%%%%%%%%%%%%%%%%%
%%%%%%%%%%%%%%%%%%%%%%%%%%%%%%%%%%%%%%%%%%%%%%%%%%%%%%%%%%%%%%%%%%%%%%%%%%%%%%%%%%%%%%%%%%%%%%%%%%%%%%%%%%%%%%%%%%%%%
%%%%%%%%%%%%%%%%%%%%%%%%%%%%%%%%%%%%%%%%%%%%%%%%%%%%%%%%%%%%%%%%%%%%%%%%%%%%%%%%%%%%%%%%%%%%%%%%%%%%%%%%%%%%%%%%%%%%%
% \spacesection
\section{\textbf{Language Proficiency}}
		\cvline{\textbf{Mandarin}}{Native}
		\cvline{\textbf{English}}{Professional Level: IELTS 6.5; CET6 476/710; CET4 544/710.}
		\cvline{\textbf{Cantonese}}{Intermediate}
%%%%%%%%%%%%%%%%%%%%%%%%%%%%%%%%%%%%%%%%%%%%%%%%%%%%%%%%%%%%%%%%%%%%%%%%%%%%%%%%%%%%%%%%%%%%%%%%%%%%%%%%%%%%%%%%%%%%%
%%%%%%%%%%%%%%%%%%%%%%%%%%%%%%%%%%%%%%%%%%%%%%%%%%%%%%%%%%%%%%%%%%%%%%%%%%%%%%%%%%%%%%%%%%%%%%%%%%%%%%%%%%%%%%%%%%%%%
%%%%%%%%%%%%%%%%%%%%%%%%%%%%%%%%%%%%%%%%%%%%%%%%%%%%%%%%%%%%%%%%%%%%%%%%%%%%%%%%%%%%%%%%%%%%%%%%%%%%%%%%%%%%%%%%%%%%%
%%%%%%%%%%%%%%%%%%%%%%%%%%%%%%%%%%%%%%%%%%%%%%%%%%%%%%%%%%%%%%%%%%%%%%%%%%%%%%%%%%%%%%%%%%%%%%%%%%%%%%%%%%%%%%%%%%%%%
%%%%%%%%%%%%%%%%%%%%%%%%%%%%%%%%%%%%%%%%%%%%%%%%%%%%%%%%%%%%%%%%%%%%%%%%%%%%%%%%%%%%%%%%%%%%%%%%%%%%%%%%%%%%%%%%%%%%%
\section{\textbf{Skills}}
		\cvline{\textbf{Technical}}{Python, PyTorch, Tensorflow + Keras, {\LaTeX}, C/C++, MATLAB}
		\cvline{\textbf{Other}}{Linux (Ubuntu), Microsoft Office, Adobe Photoshop}
%%%%%%%%%%%%%%%%%%%%%%%%%%%%%%%%%%%%%%%%%%%%%%%%%%%%%%%%%%%%%%%%%%%%%%%%%%%%%%%%%%%%%%%%%%%%%%%%%%%%%%%%%%%%%%%%%%%%%
%%%%%%%%%%%%%%%%%%%%%%%%%%%%%%%%%%%%%%%%%%%%%%%%%%%%%%%%%%%%%%%%%%%%%%%%%%%%%%%%%%%%%%%%%%%%%%%%%%%%%%%%%%%%%%%%%%%%%
%%%%%%%%%%%%%%%%%%%%%%%%%%%%%%%%%%%%%%%%%%%%%%%%%%%%%%%%%%%%%%%%%%%%%%%%%%%%%%%%%%%%%%%%%%%%%%%%%%%%%%%%%%%%%%%%%%%%%
%%%%%%%%%%%%%%%%%%%%%%%%%%%%%%%%%%%%%%%%%%%%%%%%%%%%%%%%%%%%%%%%%%%%%%%%%%%%%%%%%%%%%%%%%%%%%%%%%%%%%%%%%%%%%%%%%%%%%
%%%%%%%%%%%%%%%%%%%%%%%%%%%%%%%%%%%%%%%%%%%%%%%%%%%%%%%%%%%%%%%%%%%%%%%%%%%%%%%%%%%%%%%%%%%%%%%%%%%%%%%%%%%%%%%%%%%%%
% \spacesection
\section{\textbf{Interest}}
		\cvline{}{Artificial Intelligence, Mathematics, Physics}
%%%%%%%%%%%%%%%%%%%%%%%%%%%%%%%%%%%%%%%%%%%%%%%%%%%%%%%%%%%%%%%%%%%%%%%%%%%%%%%%%%%%%%%%%%%%%%%%%%%%%%%%%%%%%%%%%%%%%
%%%%%%%%%%%%%%%%%%%%%%%%%%%%%%%%%%%%%%%%%%%%%%%%%%%%%%%%%%%%%%%%%%%%%%%%%%%%%%%%%%%%%%%%%%%%%%%%%%%%%%%%%%%%%%%%%%%%%
%%%%%%%%%%%%%%%%%%%%%%%%%%%%%%%%%%%%%%%%%%%%%%%%%%%%%%%%%%%%%%%%%%%%%%%%%%%%%%%%%%%%%%%%%%%%%%%%%%%%%%%%%%%%%%%%%%%%%
%%%%%%%%%%%%%%%%%%%%%%%%%%%%%%%%%%%%%%%%%%%%%%%%%%%%%%%%%%%%%%%%%%%%%%%%%%%%%%%%%%%%%%%%%%%%%%%%%%%%%%%%%%%%%%%%%%%%%
%%%%%%%%%%%%%%%%%%%%%%%%%%%%%%%%%%%%%%%%%%%%%%%%%%%%%%%%%%%%%%%%%%%%%%%%%%%%%%%%%%%%%%%%%%%%%%%%%%%%%%%%%%%%%%%%%%%%%
% \spacesection
\section{\textbf{Awards}}
	\cvline{\textbf{2014 -- 2018}}
		{Four-time recipient of the Third-Class Merit Scholarship for Academic Excellence at LZU.}
	\cvline{\textbf{2019 -- 2022}}
		{Three-time recipient of the Third-Class Merit Scholarship for Academic Excellence at NKU.}
%%%%%%%%%%%%%%%%%%%%%%%%%%%%%%%%%%%%%%%%%%%%%%%%%%%%%%%%%%%%%%%%%%%%%%%%%%%%%%%%%%%%%%%%%%%%%%%%%%%%%%%%%%%%%%%%%%%%%
%%%%%%%%%%%%%%%%%%%%%%%%%%%%%%%%%%%%%%%%%%%%%%%%%%%%%%%%%%%%%%%%%%%%%%%%%%%%%%%%%%%%%%%%%%%%%%%%%%%%%%%%%%%%%%%%%%%%%
%%%%%%%%%%%%%%%%%%%%%%%%%%%%%%%%%%%%%%%%%%%%%%%%%%%%%%%%%%%%%%%%%%%%%%%%%%%%%%%%%%%%%%%%%%%%%%%%%%%%%%%%%%%%%%%%%%%%%
%%%%%%%%%%%%%%%%%%%%%%%%%%%%%%%%%%%%%%%%%%%%%%%%%%%%%%%%%%%%%%%%%%%%%%%%%%%%%%%%%%%%%%%%%%%%%%%%%%%%%%%%%%%%%%%%%%%%%
%%%%%%%%%%%%%%%%%%%%%%%%%%%%%%%%%%%%%%%%%%%%%%%%%%%%%%%%%%%%%%%%%%%%%%%%%%%%%%%%%%%%%%%%%%%%%%%%%%%%%%%%%%%%%%%%%%%%%
\section{\textbf{References}}
	% \spacesubsection
	\subsection{Prof.~Zepeng~Li}
		\cvline{}
		{Associate Professor in Machine Learning at Lanzhou University}
		\cvline{\textbf{Address}}
		{222 South Tianshui Road, Lanzhou 730000, Gansu Province, P.R. China}
		% \cvline{\textbf{Tel}}{+86 22-23503633}
		\cvline{\textbf{E-mail}}{lizp@lzu.edu.cn}
	% \subsection{\textbf{Prof.~Yao~Lu}}
	% 	\cvline{}
	% 	{Professor of Medical Image Analysis at Sun Yat-sen University}
	% 	\cvline{\textbf{Address}}
	% 	{Xingang West Road, Haizhu District, Guangzhou, Guangdong, China}
	% 	\cvline{\textbf{Tel}}{+86 18664509325}
	% 	\cvline{\textbf{E-mail}}{luyao23@mail.sysu.edu.cn}
	% \spacesubsection
	\subsection{Prof.~Yunhua~Xue}
		\cvline{}
		{Associate Professor in Computational Mathematics at Nankai University}
		\cvline{\textbf{Address}}
		{94 Weijin Road, Nankai District, Tianjin, P.R. China 300071}
		% \cvline{\textbf{Tel}}{+86 22-23503633}
		\cvline{\textbf{E-mail}}{yhxue@nankai.edu.cn}
	\subsection{Dr.~Yuanpin~Zhou}
		\cvline{}
		{Postdoctoral researcher in Medical Image Analysis at Zhejiang University}
		\cvline{\textbf{Address}}
		{866 Yuhangtang Rd, Hangzhou 310058, P.R. China}
		% \cvline{\textbf{Tel}}{+86 22-23503633}
		\cvline{\textbf{E-mail}}{zhouyp6@mail2.sysu.edu.cn}
	% \spacesubsection
	% \subsection{\textbf{Prof.~Chunlin~Wu}}
	% 	\cvline{}
	% 	{Professor of Computational Mathematics at Nankai University}
	% 	\cvline{\textbf{Address}}
	% 	{Weijin Road, Nankai District, Tianjin, China}
	% 	\cvline{\textbf{Tel}}{+86 22-23503509}
	% 	\cvline{\textbf{E-mail}}{wucl@nankai.edu.cn}
% \nocite{*}
% \printbibliography[heading=none, env=midbib, keyword=academic]

% \ \\

% {\color{color1}\textbf{Fictional Writings}}
% \printbibliography[heading=none, env=midbib, keyword=fiction]



%=============================
% this part is a simple cover letter
\clearpage
\end{document}
