\documentclass[11pt,a4paper, final]{moderncv}
\usepackage{color}
\usepackage{fontspec}
\moderncvtheme{classic}
\definecolor{color1}{RGB}{1, 52, 64}
\usepackage[scale=0.95]{geometry}
\setlength{\hintscolumnwidth}{3.5cm} 
\AtBeginDocument{\recomputelengths} 
\usepackage{xunicode}
\usepackage{xltxtra}
\usepackage[utf8]{inputenc}
\usepackage{lipsum} 
\usepackage{tikz}
\usepackage{verbatim}
\usepackage[
	sorting=none,
	minbibnames=8,
	maxbibnames=9,
	block=space
]{biblatex}
\bibliography{publications}
\defbibenvironment{midbib}
{\list
	{}
	{
		\setlength{\leftmargin}{0mm}
		\setlength{\itemindent}{-\leftmargin}
		\setlength{\itemsep}{\bibitemsep}
		\setlength{\parsep}{\bibparsep}}
	}
	{\endlist}
{\item}
\DeclareSourcemap{
	\maps[datatype=bibtex]{
		\map{
			\step[fieldsource=mydoi]
			\step[fieldset=usera, origfieldval]
		}
		\map{
			\step[fieldsource=mypdf]
			\step[fieldset=userb, origfieldval]
		}
	}
}
\DeclareFieldFormat{usera}{\color{color1}[\href{#1}{\textsc{doi}}]}
\AtEveryBibitem{
	\csappto{blx@bbx@\thefield{entrytype}}{% 
		\iffieldundef{usera}{ 
			% this gets invoked, once nothing is supplied
			% via the mypdf or mydoi value.
			% you could e.g. display a default thing here.
		}{\space\printfield{usera}}}
}
\DeclareFieldFormat{userb}{\color{color1}[\href{#1}{\textsc{pdf}}]}
\AtEveryBibitem{
	\csappto{blx@bbx@\thefield{entrytype}}{\iffieldundef{userb}{}{\printfield{userb}}}
}
\renewcommand*{\mkbibnamegiven}[1]{%
	\ifitemannotation{highlight}
	{\textbf{#1}}
	{#1}}

	\renewcommand*{\mkbibnamefamily}[1]{%
	\ifitemannotation{highlight}
	{\textbf{#1}}
	{#1}
}
\setmainfont[Numbers=OldStyle]{Times New Roman}
\AfterPreamble{\hypersetup{
  pdfcreator={XeLaTeX},
  pdftitle={Fictional CV of Hai Jiang}
}}
\usepackage{fontawesome5}
\firstname{Hai}
\familyname{Jiang}
\address{Shenzhen, Guangdong, China}{}
\extrainfo{
	\faPhone\hspace{0.5em}+86 19867713757\\
	{\small\faEnvelope}\hspace{0.5em}jiangh14@lzu.edu.cn
}
\newcommand{\spacesection}{\vspace{0.4cm}}
\newcommand{\spacesubsection}{\vspace{0.2cm}}
%===========================
\begin{document}
\recipient{Motivation Letter}{}
\date{\today}
\opening{To whom it may concern,}
\closing{Sincerely,}
\enclosure[Attached]{curriculum vit\ae{}}
%%%%%%%%%%%%%%%%%%%%%%%%%%%%%%%%%%%%%%%%%%%%%%%%%%%%%%%%%%%%%%%%%%%%%%%%%%%%%%%%%%%%%%%%%%%%%%%%%%%%%%%%%%%%%%%%%%%%%
%%%%%%%%%%%%%%%%%%%%%%%%%%%%%%%%%%%%%%%%%%%%%%%%%%%%%%%%%%%%%%%%%%%%%%%%%%%%%%%%%%%%%%%%%%%%%%%%%%%%%%%%%%%%%%%%%%%%%
%%%%%%%%%%%%%%%%%%%%%%%%%%%%%%%%%%%%%%%%%%%%%%%%%%%%%%%%%%%%%%%%%%%%%%%%%%%%%%%%%%%%%%%%%%%%%%%%%%%%%%%%%%%%%%%%%%%%%
%%%%%%%%%%%%%%%%%%%%%%%%%%%%%%%%%%%%%%%%%%%%%%%%%%%%%%%%%%%%%%%%%%%%%%%%%%%%%%%%%%%%%%%%%%%%%%%%%%%%%%%%%%%%%%%%%%%%%
%%%%%%%%%%%%%%%%%%%%%%%%%%%%%%%%%%%%%%%%%%%%%%%%%%%%%%%%%%%%%%%%%%%%%%%%%%%%%%%%%%%%%%%%%%%%%%%%%%%%%%%%%%%%%%%%%%%%%
\makelettertitle
I am writing to express my interest in the PhD position at your university, 
specializing in Computer Vision within Artificial Intelligence (AI). 
With a solid foundation in Deep Learning and a genuine passion for advancing AI research, 
I am confident that my academic background and research experience uniquely position me 
to contribute to and thrive within your esteemed research group. 
\ \\
\ \\
Driven by a deep-seated interest in Computer Science and Mathematics, 
I pursued my studies in Computational Mathematics in 2019 
after completing a Bachelor's degree in Information Security in China. 
During my Master's program, I focused on Deep Learning, 
particularly within the context of style imitation in handwriting. 
I contributed to a project aimed at replicating individual handwriting styles, 
specifically emulating the Chinese handwritten characters of Shiing Shen Chern from a set of approximately 220 characters. 
This project provided me with hands-on experience in Generative Adversarial Networks (GANs), 
manuscript research, data preprocessing, and code replication, 
culminating in my Master's thesis titled, \emph{“GANs based Personal style imitation of Chinese handwritten characters.”}
\ \\
\ \\
Following this, I had the opportunity to work on the Compressed Sensing MRI ADMM-Net project, 
which integrated Deep Learning with numerical approximation theory. 
Traditional algorithms often struggle with complex images, losing intricate details after multiple iterations. 
By leveraging Convolutional Neural Networks (CNNs), 
our team explored how Deep Learning could dynamically adjust parameters during iterations to preserve these details. 
Through this project, I gained valuable insights into convergent algorithm theory, 
proof construction, and the synergy between theory and practical application. 
This experience deepened my understanding of algorithmic approaches and refined my programming skills. 
\ \\
\ \\
After earning my Master's degree in Computational Mathematics, 
I joined a research group at Sun Yat-sen University focused on Computer-Assisted Diagnosis, 
where I served as a Research Assistant. 
My responsibilities included contributing to a successful proposal for 
a key Science and Technology program in China (2023YFE0204300), 
preparing progress and completion reports for projects funded by 
the National Natural Science Foundation of China (Grants 81971691 and 12126610), 
and drafting technical specifications for a patent and a medical device application. 
In particular, I participated in three significant research projects addressing medical challenges 
such as diagnosing Placenta Accreta Spectrum Disorders, 
predicting metastasis in Axillary Sentinel Lymph Nodes, and assessing responses to Neoadjuvant Chemotherapy via MRI. 
Although the first two projects achieved strong experimental results, 
they highlighted areas where further technological innovation could be explored. 
Through these experiences, I honed my skills in programming languages 
such as Python, PyTorch, and TensorFlow, as well as in manuscript research.
\ \\
\ \\
My long-term goal is to pursue a career in academia, 
where I can contribute meaningfully to the advancement of knowledge in Computer Science and AI. 
Your research group's work aligns closely with my own academic and professional ambitions, 
making this PhD position an ideal step toward my aspirations. 
I am eager to join an environment that provides both the intellectual rigor and resources essential to achieving my goals.
\ \\
\ \\
I am enthusiastic about the possibility of joining your research community, 
where I hope to contribute actively while further developing my expertise. 
Thank you for considering my application. 
I look forward to the opportunity to discuss how my background, skills, and goals align with your program.

\makeletterclosing

\clearpage
%%%%%%%%%%%%%%%%%%%%%%%%%%%%%%%%%%%%%%%%%%%%%%%%%%%%%%%%%%%%%%%%%%%%%%%%%%%%%%%%%%%%%%%%%%%%%%%%%%%%%%%%%%%%%%%%%%%%%
%%%%%%%%%%%%%%%%%%%%%%%%%%%%%%%%%%%%%%%%%%%%%%%%%%%%%%%%%%%%%%%%%%%%%%%%%%%%%%%%%%%%%%%%%%%%%%%%%%%%%%%%%%%%%%%%%%%%%
%%%%%%%%%%%%%%%%%%%%%%%%%%%%%%%%%%%%%%%%%%%%%%%%%%%%%%%%%%%%%%%%%%%%%%%%%%%%%%%%%%%%%%%%%%%%%%%%%%%%%%%%%%%%%%%%%%%%%
%%%%%%%%%%%%%%%%%%%%%%%%%%%%%%%%%%%%%%%%%%%%%%%%%%%%%%%%%%%%%%%%%%%%%%%%%%%%%%%%%%%%%%%%%%%%%%%%%%%%%%%%%%%%%%%%%%%%%
%%%%%%%%%%%%%%%%%%%%%%%%%%%%%%%%%%%%%%%%%%%%%%%%%%%%%%%%%%%%%%%%%%%%%%%%%%%%%%%%%%%%%%%%%%%%%%%%%%%%%%%%%%%%%%%%%%%%%
\maketitle
%%%%%%%%%%%%%%%%%%%%%%%%%%%%%%%%%%%%%%%%%%%%%%%%%%%%%%%%%%%%%%%%%%%%%%%%%%%%%%%%%%%%%%%%%%%%%%%%%%%%%%%%%%%%%%%%%%%%%
%%%%%%%%%%%%%%%%%%%%%%%%%%%%%%%%%%%%%%%%%%%%%%%%%%%%%%%%%%%%%%%%%%%%%%%%%%%%%%%%%%%%%%%%%%%%%%%%%%%%%%%%%%%%%%%%%%%%%
%%%%%%%%%%%%%%%%%%%%%%%%%%%%%%%%%%%%%%%%%%%%%%%%%%%%%%%%%%%%%%%%%%%%%%%%%%%%%%%%%%%%%%%%%%%%%%%%%%%%%%%%%%%%%%%%%%%%%
%%%%%%%%%%%%%%%%%%%%%%%%%%%%%%%%%%%%%%%%%%%%%%%%%%%%%%%%%%%%%%%%%%%%%%%%%%%%%%%%%%%%%%%%%%%%%%%%%%%%%%%%%%%%%%%%%%%%%
%%%%%%%%%%%%%%%%%%%%%%%%%%%%%%%%%%%%%%%%%%%%%%%%%%%%%%%%%%%%%%%%%%%%%%%%%%%%%%%%%%%%%%%%%%%%%%%%%%%%%%%%%%%%%%%%%%%%%
\section{\textbf{Education}}
	\cventry{\textbf{2019--2022}}{Master of Science in Computational Mathematics}{\textbf{Nankai University}}{\textbf{Tianjin}}{\textbf{China}}{}
	\cvline{\textbf{Thesis}}
	{\emph{GANs based Personal style imitation of Chinese handwritten characters}}
	\cvline{\textbf{Advisors}}
	{Prof.~Yunhua~Xue, Prof.~Chunlin~Wu}
	\cvline{\textbf{Related Courses}}
	{Approximation Theory and Methods, Numerical Optimization, Convex Analysis, 
	Variational Analysis, Real Analysis, Functional Analysis, Matrix Computation, 
	Foundations of Measure Theory and Probability, Numerical Solutions of Partial Differential Equations, and more.}
	\cvline{\textbf{Cumulative GPA}}
	{2.95/4.00}
	\cventry{\textbf{2014--2018}}{Bachelor of Engineering in Information Security}{\textbf{Lanzhou University}}{\textbf{Lanzhou}}{\textbf{China}}{}
	\cvline{\textbf{Thesis}}
	{\emph{Improved Upper Bounds of Roman Domination Number in Maximal Outerplanar Graphs}}
	\cvline{\textbf{Advisor}}
	{Prof.~Zepeng~Li}
	\cvline{\textbf{Related Courses}}
	{Discrete Mathematics, Operating Systems, Data Structures, C and C++ Programming Lab, 
	Java Programming Lab, Database Theory and Lab, Computer Organization and Design, and more.}
	\cvline{\textbf{Cumulative GPA}}
	{4.15/5.00}
%%%%%%%%%%%%%%%%%%%%%%%%%%%%%%%%%%%%%%%%%%%%%%%%%%%%%%%%%%%%%%%%%%%%%%%%%%%%%%%%%%%%%%%%%%%%%%%%%%%%%%%%%%%%%%%%%%%%%
%%%%%%%%%%%%%%%%%%%%%%%%%%%%%%%%%%%%%%%%%%%%%%%%%%%%%%%%%%%%%%%%%%%%%%%%%%%%%%%%%%%%%%%%%%%%%%%%%%%%%%%%%%%%%%%%%%%%%
%%%%%%%%%%%%%%%%%%%%%%%%%%%%%%%%%%%%%%%%%%%%%%%%%%%%%%%%%%%%%%%%%%%%%%%%%%%%%%%%%%%%%%%%%%%%%%%%%%%%%%%%%%%%%%%%%%%%%
%%%%%%%%%%%%%%%%%%%%%%%%%%%%%%%%%%%%%%%%%%%%%%%%%%%%%%%%%%%%%%%%%%%%%%%%%%%%%%%%%%%%%%%%%%%%%%%%%%%%%%%%%%%%%%%%%%%%%
%%%%%%%%%%%%%%%%%%%%%%%%%%%%%%%%%%%%%%%%%%%%%%%%%%%%%%%%%%%%%%%%%%%%%%%%%%%%%%%%%%%%%%%%%%%%%%%%%%%%%%%%%%%%%%%%%%%%%
\section{\textbf{Research Experience II}, Sun Yat-sen University}
	\cventry{\textbf{11.2022--07.2024}}{Research Assistant}{Computational Medical Image Laboratory}
	{}{}{School of Computer Science and Engineering}
	\cvline{Programme}
	{China Department of Science and Technology Key Grant, focused on Breast Cancer, 
	aims to develop models with clinical interpretability and generalization.}
	\cvline{Supervisors}{Prof.~Yao~Lu, Dr.~Ting~Song}
	\cvline{Project Focus}{Placenta Accreta Spectrum Disorders, T2-WI MRI, Prenatal Diagnosis, Multi-class classification.}
	\cvline{Experience and Skills}{Literature research, data preprocessing, model building (programming), research paper writing.}
	\cvline{Publication}
	{Submitted to ISBI 2025; Under Review; 
	\emph{“Anatomy-guided Multitask Learning for MRI-based Classification of Placenta Accreta Spectrum and its Subtypes.”}}
    \spacesubsection
    % \draw[dotted]
    % \begin{tikzpicture}
    %     \draw[dotted]
    % \end{tikzpicture}
	\cventry{\textbf{12.2023--01.2024}}{Research Assistant}{Computational Medical Image Laboratory}
	{}{}{School of Computer Science and Engineering}
	\cvline{Programme}
	{National Natural Science Foundation of China, focused on Breast Cancer, 
	aimed to develop a prediction model for the Chinese female population mainly with FFDM and US.}
	\cvline{Supervisors}{Prof.~Yao~Lu, Dr.~Xiang~Zhang}
	\cvline{Project Focus}{Breast Cancer, Dual-Energy CT, Sentinel Lymph Nodes, Metestatic status, Multi-class classification.}
	\cvline{Experience and Skills}
	{The first comprehensive research experience encompassed conducting literature reviews, designing experiments, 
	writing research papers, and working with the TensorFlow framework.}
	\cvline{Publication}
	{Submitted to MICCAI2024; in Revising; 
	\emph{“Space-Squeeze Method for Multi-Class Classification of Metastatic Lymph Nodes in Breast Cancer.”}}
% \subsection{Multi-view based Atypical Architectural Distortion Detection of Breast Cancer}
% \cventry{\textbf{08.2022--03.2023}}
% {Advisor: Yao Lu, Cooperator: Gary Pan}
% {Sun Yat-sen University}
% {Submitted to the Journal of Physics in Medicine and Biology}{Accepted}
% {
% ■	A task of the project National Natural Science Foundation of China(NSFC) under Grant 12126610, is in progress. 
% The project aimed to develop a personalized prediction model for early breast cancer prevention in the Chinese female population. \\
% ■	The task aims to detect the breast architectural distortion based on the DBT(Digital Breast Tomosynthesis) volume with two different angles. 
% Inspired by the Siamese-Networks initially used in face recognition, the task utilized the Siamese architecture to extract and compare information from both views. 
% Experimental results show that the corresponding triplet module did help the False-Positive reduction from two angles.\\
% ■	My involvement in the project focused on preparatory works including providing ideas and consultation.\\
% ■	I gained my knowledge in multiple machine learning methods, 
% including automatic assessment of architectural distortion, detection technology in deep learning, et al.\\
% }
%%%%%%%%%%%%%%%%%%%%%%%%%%%%%%%%%%%%%%%%%%%%%%%%%%%%%%%%%%%%%%%%%%%%%%%%%%%%%%%%%%%%%%%%%%%%%%%%%%%%%%%%%%%%%%%%%%%%%
%%%%%%%%%%%%%%%%%%%%%%%%%%%%%%%%%%%%%%%%%%%%%%%%%%%%%%%%%%%%%%%%%%%%%%%%%%%%%%%%%%%%%%%%%%%%%%%%%%%%%%%%%%%%%%%%%%%%%
%%%%%%%%%%%%%%%%%%%%%%%%%%%%%%%%%%%%%%%%%%%%%%%%%%%%%%%%%%%%%%%%%%%%%%%%%%%%%%%%%%%%%%%%%%%%%%%%%%%%%%%%%%%%%%%%%%%%%
%%%%%%%%%%%%%%%%%%%%%%%%%%%%%%%%%%%%%%%%%%%%%%%%%%%%%%%%%%%%%%%%%%%%%%%%%%%%%%%%%%%%%%%%%%%%%%%%%%%%%%%%%%%%%%%%%%%%%
%%%%%%%%%%%%%%%%%%%%%%%%%%%%%%%%%%%%%%%%%%%%%%%%%%%%%%%%%%%%%%%%%%%%%%%%%%%%%%%%%%%%%%%%%%%%%%%%%%%%%%%%%%%%%%%%%%%%%
\section{\textbf{Research Experience I}, Nankai University}
	\cventry{\textbf{01.2022--06.2022}}{Research Student}{Image Analysis Team}
	{}{}{School of Mathematics and Sciences}
	\cvline{Project}
	{ADMM model from the manuscript “Deep ADMM-Net for Compressed-Sensing MRI.”}
	\cvline{Supervisors}{Prof.~Chunlin~Wu, Prof.~Yunhua~Xue}
	\cvline{Focus}{Compressed-sensing Theory, Iterative Equations, Neural Networks, MRI reconstruction.}
	\cvline{Experience and Skills}
	{The second programming experience involved proving mathematical equations and applying Deep Learning techniques. 
	I reproduced the iterative mathematical equations using C++, Python, and PyTorch.}
    \spacesubsection
	\cventry{\textbf{01.2021--04.2021}}{Research Student}{Image Analysis Team}
	{}{}{School of Mathematics and Sciences}
	\cvline{Project}
	{ROF-model from the manuscript “Nonlinear Total Variation Based Noise Removal Algorithms.”}
	\cvline{Supervisor}{Prof.~Yunhua~Xue}
	\cvline{Focus}{Image Restoration, Denoise, PDE, Total-Variation Penalty.}
	\cvline{Experience and Skills}
	{Numerical experiments used both C++ and Python to build the ROF model.}
%%%%%%%%%%%%%%%%%%%%%%%%%%%%%%%%%%%%%%%%%%%%%%%%%%%%%%%%%%%%%%%%%%%%%%%%%%%%%%%%%%%%%%%%%%%%%%%%%%%%%%%%%%%%%%%%%%%%%
%%%%%%%%%%%%%%%%%%%%%%%%%%%%%%%%%%%%%%%%%%%%%%%%%%%%%%%%%%%%%%%%%%%%%%%%%%%%%%%%%%%%%%%%%%%%%%%%%%%%%%%%%%%%%%%%%%%%%
%%%%%%%%%%%%%%%%%%%%%%%%%%%%%%%%%%%%%%%%%%%%%%%%%%%%%%%%%%%%%%%%%%%%%%%%%%%%%%%%%%%%%%%%%%%%%%%%%%%%%%%%%%%%%%%%%%%%%
%%%%%%%%%%%%%%%%%%%%%%%%%%%%%%%%%%%%%%%%%%%%%%%%%%%%%%%%%%%%%%%%%%%%%%%%%%%%%%%%%%%%%%%%%%%%%%%%%%%%%%%%%%%%%%%%%%%%%
%%%%%%%%%%%%%%%%%%%%%%%%%%%%%%%%%%%%%%%%%%%%%%%%%%%%%%%%%%%%%%%%%%%%%%%%%%%%%%%%%%%%%%%%%%%%%%%%%%%%%%%%%%%%%%%%%%%%%
% \newpage
% \spacesection
\section{\textbf{Other Work Experience}}
	\subsection{\textbf{Funding}}
		\cvline{\textbf{Proposal}}
		{Proposal Writing; Accepted; China Department of Science and Technology Key Grant 2023YFE0204300.}
		\cvline{\textbf{Report}}
		{Finished two Completion Reports and two Progress Reports; Succeeded; NSFC Grant 81971697, 12126610.}
		% \spacesubsection
	\subsection{\textbf{Specification}}
		\cvline{\textbf{Patent}}
		{1 Patent Application Specification; under review.}
		\cvline{\textbf{Device}}
		{1 Medical Device Application Specification; under review.}
		% \spacesubsection
	\subsection{\textbf{Teaching Assistant}}
		\cvline{\textbf{Courses}}{Calculus; Mathematical Analysis}
		\cvline{\textbf{Thesis}}{\emph{Breast Cancer Classification Method Based on Dual-Energy CT Images}}
%%%%%%%%%%%%%%%%%%%%%%%%%%%%%%%%%%%%%%%%%%%%%%%%%%%%%%%%%%%%%%%%%%%%%%%%%%%%%%%%%%%%%%%%%%%%%%%%%%%%%%%%%%%%%%%%%%%%%
%%%%%%%%%%%%%%%%%%%%%%%%%%%%%%%%%%%%%%%%%%%%%%%%%%%%%%%%%%%%%%%%%%%%%%%%%%%%%%%%%%%%%%%%%%%%%%%%%%%%%%%%%%%%%%%%%%%%%
%%%%%%%%%%%%%%%%%%%%%%%%%%%%%%%%%%%%%%%%%%%%%%%%%%%%%%%%%%%%%%%%%%%%%%%%%%%%%%%%%%%%%%%%%%%%%%%%%%%%%%%%%%%%%%%%%%%%%
%%%%%%%%%%%%%%%%%%%%%%%%%%%%%%%%%%%%%%%%%%%%%%%%%%%%%%%%%%%%%%%%%%%%%%%%%%%%%%%%%%%%%%%%%%%%%%%%%%%%%%%%%%%%%%%%%%%%%
%%%%%%%%%%%%%%%%%%%%%%%%%%%%%%%%%%%%%%%%%%%%%%%%%%%%%%%%%%%%%%%%%%%%%%%%%%%%%%%%%%%%%%%%%%%%%%%%%%%%%%%%%%%%%%%%%%%%%
% \spacesection
\section{\textbf{Language Proficiency}}
		\cvline{\textbf{Mandarin}}{Native Speaker}
		\cvline{\textbf{English}}{IELTS 6.5; CET6 476/710; CET4 544/710; Fluent(speaking, reading, writing).}
		\cvline{\textbf{Cantonese}}{Intermediate}
%%%%%%%%%%%%%%%%%%%%%%%%%%%%%%%%%%%%%%%%%%%%%%%%%%%%%%%%%%%%%%%%%%%%%%%%%%%%%%%%%%%%%%%%%%%%%%%%%%%%%%%%%%%%%%%%%%%%%
%%%%%%%%%%%%%%%%%%%%%%%%%%%%%%%%%%%%%%%%%%%%%%%%%%%%%%%%%%%%%%%%%%%%%%%%%%%%%%%%%%%%%%%%%%%%%%%%%%%%%%%%%%%%%%%%%%%%%
%%%%%%%%%%%%%%%%%%%%%%%%%%%%%%%%%%%%%%%%%%%%%%%%%%%%%%%%%%%%%%%%%%%%%%%%%%%%%%%%%%%%%%%%%%%%%%%%%%%%%%%%%%%%%%%%%%%%%
%%%%%%%%%%%%%%%%%%%%%%%%%%%%%%%%%%%%%%%%%%%%%%%%%%%%%%%%%%%%%%%%%%%%%%%%%%%%%%%%%%%%%%%%%%%%%%%%%%%%%%%%%%%%%%%%%%%%%
%%%%%%%%%%%%%%%%%%%%%%%%%%%%%%%%%%%%%%%%%%%%%%%%%%%%%%%%%%%%%%%%%%%%%%%%%%%%%%%%%%%%%%%%%%%%%%%%%%%%%%%%%%%%%%%%%%%%%
\section{\textbf{Skills}}
		\cvline{\textbf{Technical}}{Python, PyTorch, Tensorflow, C/C++, {\LaTeX}, MATLAB, Mathematics}
		\cvline{\textbf{Other}}{Linux (Ubuntu), Microsoft Office, Adobe Photoshop}
%%%%%%%%%%%%%%%%%%%%%%%%%%%%%%%%%%%%%%%%%%%%%%%%%%%%%%%%%%%%%%%%%%%%%%%%%%%%%%%%%%%%%%%%%%%%%%%%%%%%%%%%%%%%%%%%%%%%%
%%%%%%%%%%%%%%%%%%%%%%%%%%%%%%%%%%%%%%%%%%%%%%%%%%%%%%%%%%%%%%%%%%%%%%%%%%%%%%%%%%%%%%%%%%%%%%%%%%%%%%%%%%%%%%%%%%%%%
%%%%%%%%%%%%%%%%%%%%%%%%%%%%%%%%%%%%%%%%%%%%%%%%%%%%%%%%%%%%%%%%%%%%%%%%%%%%%%%%%%%%%%%%%%%%%%%%%%%%%%%%%%%%%%%%%%%%%
%%%%%%%%%%%%%%%%%%%%%%%%%%%%%%%%%%%%%%%%%%%%%%%%%%%%%%%%%%%%%%%%%%%%%%%%%%%%%%%%%%%%%%%%%%%%%%%%%%%%%%%%%%%%%%%%%%%%%
%%%%%%%%%%%%%%%%%%%%%%%%%%%%%%%%%%%%%%%%%%%%%%%%%%%%%%%%%%%%%%%%%%%%%%%%%%%%%%%%%%%%%%%%%%%%%%%%%%%%%%%%%%%%%%%%%%%%%
% \spacesection
\section{\textbf{Interest}}
		\cvline{}{Artificial Intelligence; Mathematics; Physics}
%%%%%%%%%%%%%%%%%%%%%%%%%%%%%%%%%%%%%%%%%%%%%%%%%%%%%%%%%%%%%%%%%%%%%%%%%%%%%%%%%%%%%%%%%%%%%%%%%%%%%%%%%%%%%%%%%%%%%
%%%%%%%%%%%%%%%%%%%%%%%%%%%%%%%%%%%%%%%%%%%%%%%%%%%%%%%%%%%%%%%%%%%%%%%%%%%%%%%%%%%%%%%%%%%%%%%%%%%%%%%%%%%%%%%%%%%%%
%%%%%%%%%%%%%%%%%%%%%%%%%%%%%%%%%%%%%%%%%%%%%%%%%%%%%%%%%%%%%%%%%%%%%%%%%%%%%%%%%%%%%%%%%%%%%%%%%%%%%%%%%%%%%%%%%%%%%
%%%%%%%%%%%%%%%%%%%%%%%%%%%%%%%%%%%%%%%%%%%%%%%%%%%%%%%%%%%%%%%%%%%%%%%%%%%%%%%%%%%%%%%%%%%%%%%%%%%%%%%%%%%%%%%%%%%%%
%%%%%%%%%%%%%%%%%%%%%%%%%%%%%%%%%%%%%%%%%%%%%%%%%%%%%%%%%%%%%%%%%%%%%%%%%%%%%%%%%%%%%%%%%%%%%%%%%%%%%%%%%%%%%%%%%%%%%
% \spacesection
\section{\textbf{Awards}}
	\cvline{\textbf{2014 -- 2018}}
		{Four-time recipient of the Third-Class Merit Scholarship for Academic Excellence at Lanzhou University}
	\cvline{\textbf{2019 -- 2022}}
		{Three-time recipient of the Third-Class Merit Scholarship for Academic Excellence at Nankai University}
%%%%%%%%%%%%%%%%%%%%%%%%%%%%%%%%%%%%%%%%%%%%%%%%%%%%%%%%%%%%%%%%%%%%%%%%%%%%%%%%%%%%%%%%%%%%%%%%%%%%%%%%%%%%%%%%%%%%%
%%%%%%%%%%%%%%%%%%%%%%%%%%%%%%%%%%%%%%%%%%%%%%%%%%%%%%%%%%%%%%%%%%%%%%%%%%%%%%%%%%%%%%%%%%%%%%%%%%%%%%%%%%%%%%%%%%%%%
%%%%%%%%%%%%%%%%%%%%%%%%%%%%%%%%%%%%%%%%%%%%%%%%%%%%%%%%%%%%%%%%%%%%%%%%%%%%%%%%%%%%%%%%%%%%%%%%%%%%%%%%%%%%%%%%%%%%%
%%%%%%%%%%%%%%%%%%%%%%%%%%%%%%%%%%%%%%%%%%%%%%%%%%%%%%%%%%%%%%%%%%%%%%%%%%%%%%%%%%%%%%%%%%%%%%%%%%%%%%%%%%%%%%%%%%%%%
%%%%%%%%%%%%%%%%%%%%%%%%%%%%%%%%%%%%%%%%%%%%%%%%%%%%%%%%%%%%%%%%%%%%%%%%%%%%%%%%%%%%%%%%%%%%%%%%%%%%%%%%%%%%%%%%%%%%%
\section{\textbf{References}}
	% \spacesubsection
	\subsection{\textbf{Prof.~Zepeng~Li}}
		\cvline{}
		{Associate Professor of Machine Learning at Lanzhou University}
		\cvline{\textbf{Address}}
		{Tianshui South Road, Chengguan District, Lanzhou, Gansu, China}
		\cvline{\textbf{Tel}}{+86 22-23503633}
		\cvline{\textbf{E-mail}}{lizp@lzu.edu.cn}
	% \subsection{\textbf{Prof.~Yao~Lu}}
	% 	\cvline{}
	% 	{Professor of Medical Image Analysis at Sun Yat-sen University}
	% 	\cvline{\textbf{Address}}
	% 	{Xingang West Road, Haizhu District, Guangzhou, Guangdong, China}
	% 	\cvline{\textbf{Tel}}{+86 18664509325}
	% 	\cvline{\textbf{E-mail}}{luyao23@mail.sysu.edu.cn}
	% \spacesubsection
	\subsection{\textbf{Prof.~Yunhua~Xue}}
		\cvline{}
		{Associate Professor of Computational Mathematics at Nankai University}
		\cvline{\textbf{Address}}
		{Weijin Road, Nankai District, Tianjin, China}
		\cvline{\textbf{Tel}}{+86 22-23503633}
		\cvline{\textbf{E-mail}}{yhxue@nankai.edu.cn}
	% \spacesubsection
	% \subsection{\textbf{Prof.~Chunlin~Wu}}
	% 	\cvline{}
	% 	{Professor of Computational Mathematics at Nankai University}
	% 	\cvline{\textbf{Address}}
	% 	{Weijin Road, Nankai District, Tianjin, China}
	% 	\cvline{\textbf{Tel}}{+86 22-23503509}
	% 	\cvline{\textbf{E-mail}}{wucl@nankai.edu.cn}
% \nocite{*}
% \printbibliography[heading=none, env=midbib, keyword=academic]

% \ \\

% {\color{color1}\textbf{Fictional Writings}}
% \printbibliography[heading=none, env=midbib, keyword=fiction]



%=============================
% this part is a simple cover letter
\clearpage
\end{document}
